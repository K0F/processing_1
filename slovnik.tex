\newglossaryentry{sin()}{
	name={sin()},
	description={Vypočítáva sinus z daného úhlu. Tato funkce očekává údaj úhlu zadaný ve stupních radiánu, tedy hodnoty od 0 do 6.28. Zpětně funkce vrací hodnotu v rozmezí od -1 do 1.}
}
\newglossaryentry{vertex()}{
	name={vertex()},
	description={Funkcí {\em vertex()} definujeme body pro tvorbu složitějších tvarů. Funkci musí předcházet příkaz \vyraz{beginShape()} a musí ji následovat příkaz \vyraz{endShape()}. Přičemž záleží na pořadí, ve kterém jsou jednotlivé body definovány. Body mohou být dvourozměrné i trojrozměrné. {\em Vertex} tedy můžeme definovat pomocí souřadnic $X$, $Y$ nebo $X$, $Y$, $Z$.}
}
\newglossaryentry{noise()}{
	name={noise()},
	description={Udává hodnotu Perlinova šumu na specifické pozici. Perlinův šum je generátor pseudonáhodných čísel vytvářející plynule přecházející pseudonáhodnou hodnotu. Algoritmus byl vyvinut Kenem Perlinem v osmdesátých letech dvacátého století a je velmi často používán právě v grafických aplikacích. Využívá se například při tvorbě procedurálních textur, pohybu objektů nebo proměně geometrie v čase. Rozsah perlinova šumu se pohybuje vždy v rozmezí hodnot $0.0$ a $1.0$. Funkce {\em noise()} přijímá jednu, dvě nebo tři hodnoty. Výsledná hodnota se pohybuje ve jednorozměrném až ve trojrozměrném poli. Šum může být například snadno animován přírůstkem v parametru. Hodnoty mají tu vlastnost, že plynule přecházejí jedna ve druhou.}
}
\newglossaryentry{noFill()}{
	name={noFill()},
	description={Zakazuje výplň všech následujících objektů. Jestliže je příkaz použit dohromady s příkazem  \vyraz{noStroke()}, Processing teoreticky nebude nic vykreslovat na plátno.}
}
\newglossaryentry{mouseButton}{
	name={mouseButton},
	description={Processing automaticky registruje, které tlačítko myši bylo stisknuto. Jestliže bylo stisknuto levé tlačítko, tato systémová proměnná bude mít hodnotu {\em LEFT}. V případě pravého tlačítka, bude mít proměnná hodnotu {\em RIGHT}. V případě prostředního tlačítka pak hodnotu {\em CENTER}.}
}
\newglossaryentry{mouseClicked()}{
	name={mouseClicked()},
	description={Funkce spuštěná klikem myši. Celý jeden klik myši se skládá ze stisku tlačítka a jeho následného uvolnění.}
}
\newglossaryentry{mouseDragged()}{
	name={mouseDragged()},
	description={Funkce detekující potáhnutí myši po plátně. Táhnutím myší se rozumí, podržení tlačítka myši a pohyb kurzoru.}
}
\newglossaryentry{mouseMoved()}{
	name={mouseMoved()},
	description={Funkce detekující jakýkoli pohyb myši v rámci okna programu.}
}
\newglossaryentry{mousePressed()}{
	name={mousePressed()},
	description={Funkce je spuštěna pokaždé je-li stisknuto tlačítko myši. K následné detekci tlačítka můžeme použít proměnnou \vyraz{mouseButton}.}
}
\newglossaryentry{nf()}{
	name={nf()},
	description={Nástroj na formátování čísel do patřičného tvaru v textu. Přijímá parametr celého čísla a předřazuje před něj počet nul ze druhého parametru. Tato funkce se velmi často používá k zarovnávání řad čísel.}
}
\newglossaryentry{fill()}{
	name={fill()},
	description={Funkce nastavující barvu výplně tvarů. Například spustíme-li {\textbf fill(204, 102, 0)}, veškeré objekty, kreslené níže v kódu, budou mít oranžovou barvu výplně.}
}
\newglossaryentry{stroke()}{
	name={stroke()},
	description={Funkce nastavující barvu obrysů tvarů. Například spustíme-li {\textbf stroke(255, 0, 0)}, veškeré další kontury objektů budou mít červenou barvu.}
}
\newglossaryentry{keyPressed()}{
	name={keyPressed()},
	description={Funkce spuštěna pokaždé, je-li stisknuta libovolná klávesa na klávesnici.}
}
\newglossaryentry{keyReleased()}{
	name={keyReleased()},
	description={Funkce spuštěná tehdy, uvolníme-li libovolnou stisknutou klávesu.}
}
\newglossaryentry{for}{
	name={for},
	description={Definuje smyčku, tedy opakující se sekvenci příkazů. Struktura {\em for} se zadává pomocí tří parametrů: inicializace pozice (kde smyčka začíná), ověření (konečná podmínka která smyčku ukončuje) a přírůstku (jeden krok rozdílu při opakování smyčky). Tyto části jsou vždy oddělené středníkem. Smyčka pokračuje dokud je naplněn výsledek prostřední podmínky, tedy \vyraz{true}.}
}
\newglossaryentry{frameRate()}{
	name={frameRate()},
	description={Nastavuje počet okének, které mají být vykresleny za jednu vteřinu. Výchozí nastavení je šedesát okének za vteřinu. Pokud je údaj vyšší než Processing může v danou chvíli vykreslovat, počet okének za vteřinu se sníží na maximální možnou hodnotu.}
}
\newglossaryentry{endRecord()}{
	name={endRecord()},
	description={Zastavuje nahrávání do souboru spuštěné pomocí funkce \vyraz{beginRecord()}, dále uzavírá zápis do souboru (EOF).}
}
\newglossaryentry{endShape()}{
	name={endShape()},
	description={Párová funkce s \vyraz{beginShape()}, uzavírá kreslení tvaru. Funkce může být doplněna o výraz {\em CLOSE} (psáno dovnitř kulatých závorek), který uzavře tvar a pokusí se vytvořit jeho výplň.}
}
\newglossaryentry{ellipse()}{
	name={ellipse()},
	description={Kreslí elipsu nebo kruh. Elipsa se stejnou šířkou i výškou představuje kruh. První dva parametry udávají pozici, třetí parametr uvádí šířku elipsy a čtvrtý výšku elipsy. Nastavení způsobu vytváření elips lze provést pomocí funkce \vyraz{ellipseMode()}.}
}
\newglossaryentry{else}{
	name={else},
	description={Doplňuje podmínku \vyraz{if()} -- \uv{jestliže} o \uv{jestliže ne}. Blok kódu za tímto příkazem je spuštěn jestliže není předchozí \vyraz{if()} podmínka splněna. Struktura kódu tím dovoluje spustit dva různé bloky kódu v případě naplnění podmínky anebo její nenaplnění.}
}
\newglossaryentry{image()}{
	name={image()},
	description={Zobrazuje obrázek v datatypu \vyraz{PImage}. K jejich načtení pužíváme funkci \vyraz{loadImage()}. Barvu a průhlednost obrázku lze dále definovat pomocí funkce \vyraz{tint()}. Funkce {\em image()} jako první přijímá parametr s názvem proměnné obrázku, který bude zobrazovat. Zbylé dva parametry jsou pozice vykreslení obrázku v ose $X$ a $Y$. Ve výchozím módu souřadnice určují levý horní roh obrázku. Třetí a čtvrtý parametr je nepovinný, může dále nastavit šířku a výšku zobrazovaného obrázku. Módy pro alternativní zobrazování lze přepínat pomocí funkce \vyraz{imageMode()}.}
}
\newglossaryentry{loadImage()}{
	name={loadImage()},
	description={Načítá obrázek do proměnné ve formátu \vyraz{PImage}. V kulatých závorkách se definuje cesta a název obrázku (včetně přípony) ve formátu \vyraz{String}. Obrázky by měly být umístěny v adresáři {\em DATA}.  Processing podporuje grafické formáty {\em PNG, GIF, JPEG a TGA}. Průhlednost obrázků je podporována ve formátech {\em GIF} nebo {\em PNG}.}
}
\newglossaryentry{imageMode()}{
	name={imageMode()},
	description={Ovlivňuje způsob zobrazování obrázků. Dovoluje použít konstanty {\em CORNER, CORNERS a CENTER}. Ve výchozím nastavení {\em CORNER} bude obrázek umístěn levým horním rohem na definované souřadnice $X$ a $Y$. Mód {\em CORNERS} mění chování druhých dvou parametrů, které namísto relativní šířky a výšky nastavují absolutní hodnoty v osách $X$ a $Y$. Poslední mód {\em CENTER}, umístí obrázek na střed definovaných souřadnic, zbylé dva parametry opět udávají relativní šířku a výšku obrázku.}
}
\newglossaryentry{colorMode()}{
	name={colorMode()},
	description={Funkce měnící způsob, jakým Processing interpretuje barvu. Možné argumenty jsou RGB a HSB. Nepovinný je druhý argument, určující rozsah hodnot. Výchozí nastavení této hodnoty je 255. Příkaz ovlivňuje nastavení barev pro kresbu, tj. například pro funkce \vyraz{fill()} a \vyraz{stroke()}.}
}
\newglossaryentry{color()}{
	name={color()},
	description={Vytváří barvu ve speciálním datatypu {\em color}. Vstupní hodnoty jsou interpretovány buď jako hodnoty RGB (červená, zelená, modrá), nebo HSB (odstín, sytost, jas). Jendotlivé módy vytváření barev lze přepínat funkcí \vyraz{colorMode()}. Ve výchozím nastavení mají barvy rozmezí 256 odstínu šedi, tedy konkrétně $0..255$.}
}

\newglossaryentry{size()}{
	name={size()},
	description={Nastavuje rozměry okna programu. Hodnoty jsou udávané v pixelech pro osu $X$ a $Y$. Tento příkaz by měl být spuštěn jako první ve funkci {\em setup()}. Jestliže není příkaz spuštěn, okno programu má výchozí velikost $100x100$ pixelů. Ve funkci {\em size()} je dále přípustný třetí parametr, ten je vyjádřen slovy: JAVA2D (výchozí render), P2D (akcelerovaný 2D render), P3D (akcelerovaný 3D render), OPENGL (3D akcelerovaný render vyžadující načtení externí knihovny)}
}
\newglossaryentry{smooth()}{
	name={smooth()},
	description={Zapíná takzvaný \uv{anti--aliasing}, tedy vyhlazování hran kresby. Funkce nastavuje vyhlazování pro celý program. Zapnutí vyhlazování, má za následek poměrně výrazné zpomalení kresby. Výstup kresby je ovšem o poznání hladší.}
}
\newglossaryentry{noSmooth()}{
	name={noSmooth()},
	description={Opak funkce \vyraz{smooth()}. Vypíná vyhlazování hran.}
}
\newglossaryentry{sphereDetail()}{
	name={sphereDetail()},
	description={Funkce nastavuje úroveň detailu při vykreslování objektu koule. Parametr zadaný ve formátu čísla \uv{float} ovlivňuje vykreslovaný počet stran koule.}
}
\newglossaryentry{strokeWeight()}{
	name={strokeWeight()},
	description={Funkce nastavuje šířku vykreslovaných obrysů v pixelech.}
}
\newglossaryentry{String}{
	name={String},
	description={Datatyp pro řetězec textu. Jedná se o speciální datatyp schopný uchovat více znaků pod jednou proměnnou. S tímto datatypem se pojí řada speciálních funkcí pro nakládání s textem. P5edstavit si jej můžeme jako jednorozměrné pole znaků.}
}
\newglossaryentry{println()}{
	name={println()},
	description={Funkce pro tisk do konzole. Funguje stejně jako příkaz \vyraz{print()}, s tím rozdílem, že každou tisknutou hodnotu zakončuje novým řádkem.}
}
\newglossaryentry{print()}{
	name={print()},
	description={Funkce pro tisk do konzole. Vstupní hodnotou může být jakákoli proměnná, holý text nebo kombinace obojího. Slouží jako základní nástroj ke sledování proměn v programu. Využívá se zejména pří vylaďování programu a při kontrole výsledků.}
}
\newglossaryentry{parseFloat()}{
	name={parseFloat()},
	description={Funkce pro čtení čísel ve formátu (\vyraz{String}) a získávání číselné hodnoty s desetinnou čárkou ve tvaru {\em float}.}
}
\newglossaryentry{parseInt()}{
	name={parseInt()},
	description={Funkce pro čtení čísel ve formátu (\vyraz{String}) a získávání celých čísel ve tvaru {\em int}.}
}
\newglossaryentry{this}{
	name={this},
	description={Speciální výraz vztahující se k objektu, ve kterém se zrovna pohybujeme. Slouží k identifikaci zejména pro práci knihovnami nebo námi definovanými třídami.}
}
\newglossaryentry{point()}{
	name={point()},
	description={Kreslí bod ve dvourozměrném zobrazení. Funkce vyžaduje parametry $X$, $Y$. Při trojrozměrném zobrazení bodu funkce vyžaduje tři parametry pro $X$, $Y$ a $Z$. Barvu bodu můžeme kontrolovat pomocí příkazu \vyraz{stroke()}.}
}
\newglossaryentry{triangle()}{
	name={point()},
	description={Kreslí trojúhelník. Vyžaduje parametry $X_1$, $Y_1$, $X_2$, $Y_2$, $X_3$ a $Y_3$. Barvu trojúhelníku můžeme kontrolovat pomocí příkazu \vyraz{stroke()}. Barvu výplně zadáváme pomocí funkce \vyraz{fill()}.}
}
\newglossaryentry{quad()}{
	name={quad()},
	description={Funkce kreslící čtyřúhelník. Nejvíce je použitelná pro lichoběžníky. Funkce vyžaduje parametry $X_1$, $Y_1$, $X_2$, $Y_2$, $X_3$, $Y_3$, $X_4$ a $Y_4$. Barvu čtyřúhelníku můžeme kontrolovat pomocí příkazu \vyraz{stroke()}. Barvu výplně definujeme funkcí \vyraz{fill()}.}
}
\newglossaryentry{radians()}{
	name={radians()},
	description={Převádí uhlové stupně do stupňů v radiánech. Tedy například hodnotu $0$ až $360$ převede na $0$ až $2 * \pi$}
}
\newglossaryentry{rotate()}{
	name={rotate()},
	description={Provádí rotaci plátna ve dvourozměrném zobrazení. Střed rotace je vždy aktuální souřadnice $X = 0$ a $Y = 0$.}
}
\newglossaryentry{translate()}{
	name={translate()},
	description={Provádí pohyb celého plátna na zadané souřadnice. V trojrozměrném módu vykreslování požaduje třetí rozměr pro osu $Z$}
}
\newglossaryentry{scale()}{
	name={scale()},
	description={Mění velikost geometrie vůči vykreslovacímu plátnu. Výchozí hodnota je $1.0$}
}
\newglossaryentry{pushMatrix()}{
	name={pushMatrix()},
	description={Ukládá nastavení koordinačního systému (neboli matrix). Funkce dovoluje vrstvení jednotlivých transformačních operací, jako je \vyraz{rotate()},\vyraz{translate()} nebo \vyraz{scale()}. Funkce vyžaduje následné ukončení transformace příkazem \vyraz{popMatrix()}}
}
\newglossaryentry{popMatrix()}{
	name={popMatrix()},
	description={Funkce načítá uloženého koordinačního systému a uzavírá blok transformací. Funkce vyžaduje uložení koordinátů, předchozím příkaz \vyraz{pushMatrix()}.}
}
\newglossaryentry{rotateX()}{
	name={rotateX()},
	description={Funkce provádí rotaci celým prostorem kolem osy $X$. Vyžaduje hodnotu udanou ve stupních -- radiánech. Hodnota se pohybuje v rozmezí $0$ až $2 * \pi$.}
}
\newglossaryentry{rotateY()}{
	name={rotateY()},
	description={Funkce provádí rotaci celým prostorem kolem osy $Y$. Vyžaduje hodnotu udanou ve stupních -- radiánech. Hodnota se pohybuje v rozmezí $0$ až $2 * \pi$.}
}
\newglossaryentry{rotateZ()}{
	name={rotateZ()},
	description={Funkce provádí rotaci celým prostorem kolem osy $Z$. Vyžaduje hodnotu udanou ve stupních -- radiánech. Hodnota se pohybuje v rozmezí $0$ až $2 * \pi$.}
}
\newglossaryentry{PI}{
	name={PI},
	description={Konstanta určující číslo $\pi$, 3,14159265\ldots .}
}
\newglossaryentry{HALFPI}{
	name={HALFPI},
	description={Konstanta určující polovinu čísla $\pi$, 3,14159265\ldots .}
}
\newglossaryentry{QUARTERPI}{
	name={QUARTERPI},
	description={Konstanta určující čtvrtinu číslo $\pi$, 3,14159265\ldots .}
}
\newglossaryentry{CSV}{
	name={CSV},
	description={Neboli \uv{Comma Separated Values}. Jedná se o uznávaný standart uložených dat.  S tímto standardem  zachází mnoho programů. Jedná se v podstatě o textový soubor s informacemi oddělenými specifickým znakem. Nejčastěji je tímto specifickým znakem čárka. Rozlišovacím znakem může ovšem být například středník nebo teoreticky jakýkoli jiný znak.}
}
\newglossaryentry{saveStrings()}{
	name={saveStrings()},
	description={Pomocí příkazu {\em saveStrings()} ukládáme pole ve formátu \vyraz{String}. Tento příkaz přijme dva parametry: název souboru a název našeho pole s textovou informací. Příkaz ukládá textový soubor naplněný hodnotami z pole. Jednu hodnotu zapisuje vždy na nový řádek.}
}
\newglossaryentry{save()}{
	name={save()},
	description={Pomocí příkazu save() ukládáme plátno programu do obrázku. Příkaz vyžaduje cestu k souboru. Cestu lze zadat buď úplnou (absolutní) nebo jen pouhý název souboru. Zadáním pouhého názvu se soubor uloží přímo do adresáře sketche. Příkaz dále rozlišuje mezi příponami automaticky. Obrázek tak uloží v patřičných formátech: TIF, JPG, PNG nebo TGA.}
}
\newglossaryentry{saveFrame()}{
	name={saveFrame()},
	description={Příkaz ukládá sérii obrázků. Stejně jako příkaz \slovnik{save()}, funkce vyžaduje cestu k souboru. Speciální vlastností příkazu je možnost vložit do názvu souboru symboly ($#$). Ty budou posléze detekovány a nahrazeny číslováním sekvence okének. Každé číslo bude označovat číslo okénka podle proměnné \slovnik{frameCount()}.}
}
\newglossaryentry{seed}{
	name={seed},
	description={Seed je proměnná, která obecně udává chování pseudonáhodných hodnot. Pomocí hodnoty \em{seed} v programu můžeme nastavit opakovatelnost náhodných čísel pokaždé když program spustíme. Varianta pro generovanou funkci \vyraz{random()} je \vyraz{randomSeed()}.}
}
\newglossaryentry{randomSeed()}{
	name={randomSeed()},
	description={Funkce udává takzvanou hodnotu \vyraz{seed} pro všechny následující funkce \vyraz{random()}. Nezadáme-li hodnotu \vyraz{seed}, funkce \vyraz{random()} bude pokaždé generovat neopakovatelné výsledky.}
}

\newglossaryentry{random()}{
	name={random()},
	description={Funcke, která vrací pseudonáhodné číslo. Není-li zadána žádná hodnota do kulatých závorek, výsledek funkce se pohybuje v rozmezí mezi $1..0$. Je-li zadána jedna vstupní hodnota, číslo se pohybuje mezi nulou a zadanou hodnotou. Jsou-li zadány dvě hodnoty oddělené čárkou. Výsledek se bude pohybovat v rozmezí těchto dvou hodnot.}
}

\newglossaryentry{millis()}
{
  name={millis()},
  description={Funkce udávající počet uplynulých milivteřin od startu programu. Na rozdíl od proměnné \vyraz{frameCount}, tato funkce udává precizní časový údaj od startu programu. Výsledná hodnota nezávisí na počtu vykreslených okének za vteřinu. Výsledný údaj tak lze využít například pro přesné časování animace, a to bez ohledu na počet okének za vteřinu. V případě ukládání animací do videa je pro časování animace naopak téměř nepoužitelný, především kvůli zpomalení animace při ukládání okének.}
}
\newglossaryentry{frameCount}
{ 
  name={frameCount},
  description={Proměnná uchovávající údaj o počtu vykreslených okének od startu programu.}
}
\newglossaryentry{interakce}
{
  name={interakce},
  description={{\em (lat. interactio od inter-agere, jednat mezi sebou)} znamená vzájemné působení, jednání, ovlivňování všude tam, kde se klade důraz na vzájemnost a oboustrannou aktivitu na rozdíl od jednostranného, například kauzálního působení.}
}
\newglossaryentry{mouseX}
{
  name={mouseX},
  description={Proměnná která udává pozici kurzoru na plátně Processingového okna v ose $X$.}
}
\newglossaryentry{mouseY}
{
  name={mouseY},
  description={Proměnná která udává pozici kurzoru na plátně Processingového okna v ose $Y$.}
}
\newglossaryentry{rectMode()}
{
  name={rectMode()},
  description={Funkce ovládá mód vykreslování obdélníku. Možné vstupní parametry jsou pouze čtyři:  {\em CORNER} je základní chování funkce \vyraz{rect()}. Tedy prvními dvěma parametry při kresbě umístíme nejdříve prní roh v ose $X$ a $Y$. Poté definujeme relativní šířku a výšku obdélníku. Druhou možností je {\em CORNERS}. Tento mód definuje dva rohy zvlášť, tedy v absolutních číslech na ploše. Dalším módem je {\em CENTER}. V tomto módu jsou obdélníky vystředěny kolem body $X$ a $Y$. Pomocí dvou dalších parametrů opět funkce \vyraz{rect()} definuje šířku a výšku tohoto obdélníku. Posledním módem je {\em RADIUS}. Ten nastavuje opět středové body $X$ a $Y$ v prvních dvou parametrech. Druhé dva parametry udávají polovinu šířky a polovinu výšky obdélníku.}
}

\newglossaryentry{rect()}
{
  name={rect()},
  description={Funkce vykresluje obdélník. Funkce přijímá čtyři parametry: počátek $X$, počátek $Y$, šířku a výšku. Možné způsoby vykreslování obdélníku se dají nastavit funkcí \vyraz{rectMode()}}
}

\newglossaryentry{background()}
{
  name={background()},
  description={Funkce nastavuje barvu pozadí na plátně. Standardní barva je světle šedá. Spuštěním této funkce s definicí barvy v kulatých závorkách, bude vyplněna celá plocha jednolitou barvou.}
}


\newglossaryentry{void}
{
  name={void},
  description={Neboli prázdná funkce. Z anlgického \uv{prázdno}. Funkce která zpět nevrací žádný výsledek, pouze spouští sérii zadaných příkazů. Ty jsou uvozeny složenými závorkami.}
}

\newglossaryentry{indenting}
{
  name={indenting},
  description={Neboli zarovnávání kódu do úhledných paragrafů. Zarovnávání slouží k rychlejší orientaci programátora ve struktuře kódu.}
}


\newglossaryentry{syntax highlighting}
{
  name={syntax highlighting},
  description={Barevné odlišování slov podle jejich významů. Odlišování slouží k lepší orientaci programátora v kódu.}
}

\newglossaryentry{flow}
{
  name={flow},
  description={Nebo--li plynutí, tok, je zvláštním stavem mysli popisovaným programátory. Jedná se o stav, kdy je člověk plně zanořen do práce. Jakékoli vyrušení z tohoto stavu, si vyžaduje dlouhou koncentraci k opětovnému nastolení bdělosti. Podle zkušených programátorů vyvolání takového stavu trvá několik desítek minut práce s kódem.}
}

\newglossaryentry{setup()}
{
  name={setup()},
  description={Základní funkce pro nastavení výchozích parametrů programu. Tato funkce je spuštěna jednou, vždy na začátku běhu programu.}
}

\newglossaryentry{draw()}
{
  name={draw()},
  description={Hlavní smyčka Processingu. Veškerý kód uzavřený v této funkci bude vykreslen jednou za okénko. V závislosti na náročnosti operace, se smyčka opakuje šedesátkrát za vteřinu. V této funkci nejčastěji dochází ke kresbě nebo interakci s uživatelem.}
}

\newglossaryentry{stroj}
{
  name={stroj},
  description={Stroj je technické zařízení, které přeměňuje jeden druh energie nebo síly v jiný - ať už kvalitativně nebo kvantitativně. Původně byly stroje jen mechanické, ale dnes se tak označují i zařízení pracující na jiných fyzikálních či technických principech - například elektrický transformátor. Strojem je v této knize téměř výhradně myšlen počítač. Počítač je programovatelný typ stroje který přijímá vstup, ukládá a zpracovává data a umožňuje výstup v požadovaném formátu.},
  plural={stroje}
}

\newglossaryentry{slovník}
{
  name={slovník},
  description={právě se zde nacházíte},
  plural={slovníky}
}


\newglossaryentry{true}
{
  name={true},
  description={Pravda, neboli 1.}
}

\newglossaryentry{false}
{
  name={false},
  description={Nepravda, neboli 0.}
}

\newglossaryentry{boolean}
{
  name={boolean},
  description={Datatyp který může mít jen dva stavy: \vyraz{true} a \vyraz{false}}
}

\newglossaryentry{Built with Processing}
{
	name={Built with Processing},
	description={Doslova znamená: \uv{Postaveno s Processingem}. Jedná se o zvláštní komunitní frázi, která se objevuje u projektů využívajících Processing. Fráze vyjadřuje určitý vděk všem participantům a tvůrcům Processingu, za tvorbu tohoto nástroje.}
}

\newglossaryentry{otevřený software}
{
  name={otevřený software},
  description={Také nazývaný \uv{svobodný}. Software s veřejně dostupným zdrojovým kódem.},
  plural={otevřeného softwaru}
}

\newglossaryentry{GNU / Linux}
{
  name={GNU / Linux},
  description={Zkratka \uv{Gnu Is not Unix}. {\em GNU} je projekt původně založený Richardem Stallmanem, jedná se o operační systém a rodinu programů s otevřeným zdrojovým kódem.}
}

\newglossaryentry{GNU / GPL}
{
  name={GNU / GPL},
  description={Jedna z licencí otevřeného softwaru zaručující volně dostupný zdrojový kód. Základním konceptem licence je podmínečná volná dostupnost zdrojového kódu takto licencovaného programu. A dále nutnost použití stejné licence pro veškeré projekty takový kód využívající.}
}

\newglossaryentry{MIT licence}
{
  name={indenting},
  description={Je licencí kompatibilní s licencí \vyraz{GNU / GPL}, jedná se o zvláštní licenci modifikovanou na půdě Univerzity MIT {\em --Massachusetts Institute of Technology}.},
  plural={MIT licencí}
}


\newglossaryentry{kompilace}
{
	name={kompilace},
	description={Proces překladu z čitelného textu do strojového kódu.}
}

\newglossaryentry{zdrojový kód}
{
	name={zdrojový kód},
	description={Kód který je čitelný pro člověka.},
	plural={zdrojové kódy}
}

\newglossaryentry{binarní kód}
{
	name={binární kód},
	description={Též strojový kód. Slovo vycházející z anglického slova {\em binary} -- dvojkové soustavy. V počítačové terminologii označuje kód srozumitelný pro procesor. Nejčastěji se s ním setkáme například v případě spustitelných aplikací.},
	plural={binární kódy}
}

\newglossaryentry{Java}
{
	name={Java},
	description={Java je objektově orientovaný programovací jazyk, který vyvinula firma Sun Microsystems a představila jej 23. května 1995.}
}

\newglossaryentry{sketch}
{
	name={sketch},
	description={Nebo-li {\em náčrt}, je koncept v prostředí Processing pro uchovávání jednotlivých projektů v adresářích. Jedna {\em sketch} je adresář, kam Processing ukládá data uživatele (tj. především kód a externí soubory).},
	plural={sketche}
}

\newglossaryentry{CODE}
{
	name={CODE},
	description={Adresář uvnitř \slovnikpl{sketch}, který může obsahovat externí \slovnik{Java} a \slovnik{binární kód} programy nebo knihovny třetí strany.}
}
\newglossaryentry{class}{
	name={class},
	description={Takzvaná třída. Slovo označující deklaraci nové třídy, definici objektu. Třída je souhrn funkcí a proměnných a představuje šablonu pro objekt. Jména tříd jsou většinou označovány počátečním velkým písmenem, instance objektů (produkt ze šablony) pak písmenem malým. Třída je základní stavební kámen objektově orientovaného programování.}
}
\newglossaryentry{DATA}
{
	name={DATA},
	description={Adresář uvnitř \slovnikpl{sketch}, který obsahuje další externí soubory (obrázky, textové soubory, atd.).}
}
\newglossaryentry{acos()}{
	name={acos()},
	description={Inverzní funkce \vyraz{cos()}, arkuskosinus. Funkce převádí hodnoty rozmezí od $-1.0$ do $1.0$ na hodnoty v rozmezí od $0$ do $\pi$ {\em (3.1415927)}.}
}
\newglossaryentry{atan2()}{
	name={atan2()},
	description={Vypočítává úhel z jednoho bodu do bodu druhého. K výpočtu můžeme použít zápis {\em atan2(}$Y_2-Y_1$,$X_2-X_1${\em )}. Výsledný úhel je udáván v radiánech, tedy v rozmezí od $-\pi$ do $\pi$. {\em Atan2()} je funkce, která se hojně využívá při určení směru od objektu ke kurzoru nebo jinému bodu. Pozor při zadávání parametrů. Všimněte si obráceného pořadí parametrů. Oproti zvyklosti jsou zadávány v pořadí $Y$, $X$}
}
\newglossaryentry{ellipseMode()}{
	name={ellipseMode()},
	description={Ovlivňuje způsob jakým je vykreslována elipsa. Možné módy jsou {\em RADIUS, CENTER, CORNER, CORNERS}. Výchozí módem je {\em CENTER}. První dva parametry nastavují pozici elipsy, druhé dva pak šířku a výšku elipsy. V případě módu {\em RADIUS} druhé dva parametry určují poloměry elipsy. Mód {\em CORNER} definuje pozici levého horního bodu, následně šířku a výšku. Poslední mód {\em CORNERS} definuje dva body mezi které bude elipsa vepsána.}
}
\newglossaryentry{beginRecord()}{
	name={beginRecord()},
	description={Otevírá nový soubor pro zápis do souboru ve formátu {\em PDF} nebo {\em DXF}. Kresba tak probíhá souběžně na plátno i do souboru. Pro ukončení kresby musíme spustit funkci \vyraz{endRecord()}.}
}
\newglossaryentry{beginShape()}{
	name={beginShape()},
	description={Za pomoci funkcí {\em beginShape()} a \vyraz{endShape()} můžeme definovat složitější tvary. Funkce {\em beginShape()} otevírá definici tvaru. Následuje funkce \vyraz{vertex()}, která určuje jednotlivé body tvaru. Tvar je nakonec uzavřen příkazem \vyraz{endShape()}. Kresba tvarů může probíhat ve dvou i třech rozměrech. V závislosti na renderu můžeme definovat \vyraz{vertex()} se dvěmi nebo třemi hodnotami. Doplňujícím parametrem v kulatých závokách funkce {\em beginShape()} dále může být uveden způsob, jakým budou body spojovány. Přípustné varianty jsou: {POINTS, LINES, TRIANGLES, TRIANGLE\_FAN, TRIANGLE\_STRIP, QUADS a QUAD\_STRIP}.}
}
\newglossaryentry{exit()}{
	name={exit()},
	description={Ukončí program. Program bez funkce \vyraz{draw()} bude ukončen automaticky poslední řádkou kǒdu. Programy obsahující funkci \vyraz{draw()} běží dokud není program zvenčí ukončen nebo není zavolán příkaz {\textbf exit()}.}
}
\newglossaryentry{camera()}{
	name={camera()},
	description={Nastavuje pozici kamery skrze parametry: pozice oka, střed zobrazení a osa označující směr nahoru. Funkce zavolaná bez jakýchkoli parametrů nastaví kameru na výchozí pozici. Výchozí pozice lze vyjádřit následovně: {\em camera(width/2.0, height/2.0, (height/2.0) / tan(PI*60.0 / 360.0), width/2.0, height/2.0, 0, 0, 1, 0)}.}
}
\newglossaryentry{box()}{
	name={box()},
	description={Krychle nebo kvádr. Krychli můžeme definovat jedním parametrem, který určuje délku jedné hrany. Kvádr můžeme definovat stejným příkazem zadáním délky tří hran: $X$, $Y$, $Z$.}
}
\newglossaryentry{cos()}{
	name={cos()},
	description={Počítá hodnotu kosinu z úhlu. Tato funkce očekává parametr vyjadřující úhel, parametr může být v rozsahu od $0$ do $\pi*2$. Hodnoty se pak pohybují mezi $-1$ do $1$.}
}