\newglossaryentry{flow}
{
  name={flow},
  description={nebo-li plynutí, tok, je zvláštním stavem mysli popisovaným programátory, jedná se o stav kdy je člověk plně zanořen do práce a jakékoli vyrušení z tohoto stavu si vyžaduje opětovné nastolování, podle zkušených progrmátorů nastolení takového stavu obvykle trvá 10-15 minut práce s kódem.}
}

\newglossaryentry{draw()}
{
  name={draw()},
  description={draw je kreslící funkcí Processingu, veškerý kód uzavřený v této funkci bude vykreslen jednou za okénko, v závislosti na náročnosti pak standartně šedesátkrát za vteřinu, tuto kreslící funkci je nezbytná pro animovaný výstup nebo interakci s uživatelem}
}

\newglossaryentry{stroj}
{
  name={stroj},
  description={Stroj je technické zařízení, které přeměňuje jeden druh energie nebo síly v jiný - ať už kvalitativně nebo kvantitativně. Původně byly stroje jen mechanické, ale dnes se tak označují i zařízení pracující na jiných fyzikálních či technických principech - například elektrický transformátor. Strojem je v této knize téměř výhradně myšlen počítač. Počítač je programovatelný typ stroje který přijímá vstup, ukládá a zpracovává data a umožňuje výstup v požadovaném formátu.},
  plural={stroje}
}

\newglossaryentry{výraz ve slovníku}
{
  name={výraz ve slovníku},
  description={ukázkový výraz ve slovníku},
  plural={výrazy ve slovníku}
}


\newglossaryentry{true}
{
  name={true},
  description={pravda, neboli 1}
}

\newglossaryentry{false}
{
  name={false},
  description={nepravda, neboli 0}
}

\newglossaryentry{boolean}
{
  name={boolean},
  description={datatyp který může mít jen dva stavy \vyraz{true} a \vyraz{false}}
}

\newglossaryentry{GNU / Linux}
{
  name={GNU / Linux},
  description={Gnu Is not Unix, GNU je projekt založený Richardem Stallmanem, jedná se o operační systém a rodinu programů s otevřeným zdrojovým kódem.}
}

\newglossaryentry{GNU / GPL}
{
  name={GNU / GPL},
  description={jedna z licencí otevřeného softwaru zaručující otevřenost kódu, kterou v případě dalšího použití vyžaduje i u programů, které tento kód využívají}
}


\newglossaryentry{kompilace}
{
	name={kompilace},
	description={proces překladu z čitelného textu do strojového kódu}
}


\newglossaryentry{zdrojový kód}
{
	name={zdrojový kód},
	description={kód čitelný pro člověka},
	plural={zdrojové kódy}
}

\newglossaryentry{binarní kód}
{
	name={binární kód},
	description={též strojový kód, slovo vycházející z anglického slova {\em binary} dvojkové soustavy, v počítačové terminologii označuje strojový kód, tedy kód prošlý procesem \slovnik{kompilace}.},
	plural={binární kódy}
}

\newglossaryentry{Java}
{
	name={Java},
	description={Java je objektově orientovaný programovací jazyk, který vyvinula firma Sun Microsystems a představila 23. května 1995.}
}

\newglossaryentry{sketch}
{
	name={sketch},
	description={nebo-li {\em náčrt}, je koncept v prostředí Processing uchovávání jednotlivých projektů do adresářů, jedna {\em sketch} je adresář, kam Processing ukládá data uživatele (tj. především kód a externí soubory)},
	plural={sketche}
}

\newglossaryentry{CODE}
{
	name={CODE},
	description={adresář uvnitř \slovnikpl{sketch}, který obsahuje externí \slovnik{Java} \slovnik{binární kód} programy nebo knihovny třetí strany (obrázky,textové soubory, atd.)}
}

\newglossaryentry{DATA}
{
	name={DATA},
	description={adresář uvnitř \slovnikpl{sketch}, který obsahuje vaše externí soubory (obrázky,textové soubory, atd.)}
}
