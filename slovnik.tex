\newglossaryentry{stroj}
{
  name={stroj},
  description={Stroj je technické zařízení, které přeměňuje jeden druh energie nebo síly v jiný - ať už kvalitativně nebo kvantitativně. Původně byly stroje jen mechanické, ale dnes se tak označují i zařízení pracující na jiných fyzikálních či technických principech - například elektrický transformátor. Strojem je v této knize téměř výhradně myšlen počítač. Počítač je programovatelný typ stroje který přijímá vstup, ukládá a zpracovává data a umožňuje výstup v požadovaném formátu.},
  plural={stroje}
}

\newglossaryentry{výraz ve slovníku}
{
  name={výraz ve slovníku},
  description={ukázkový výraz ve slovníku},
  plural={výrazy ve slovníku}
}


\newglossaryentry{true}
{
  name={true},
  description={pravda, neboli 1}
}

\newglossaryentry{false}
{
  name={false},
  description={nepravda, neboli 0}
}

\newglossaryentry{boolean}
{
  name={boolean},
  description={datatyp který může mít jen dva stavy \vyraz{true} a \vyraz{false}}
}

\newglossaryentry{GNU / Linux}
{
  name={GNU / Linux},
  description={Gnu Is not Unix, GNU je projekt založený Richardem Stallmanem, jedná se o operační systém a rodinu programů s otevřeným zdrojovým kódem.}
}

\newglossaryentry{GNU / GPL}
{
  name={GNU / GPL},
  description={jedna z licencí otevřeného softwaru zaručující otevřenost kódu, kterou v případě dalšího použití vyžaduje i u programů, které tento kód využívají}
}
