\newglossaryentry{interakce}
{
  name={interakce},
  description={{\em (lat. interactio od inter-agere, jednat mezi sebou)} znamená vzájemné působení, jednání, ovlivňování všude tam, kde se klade důraz na vzájemnost a oboustrannou aktivitu na rozdíl od jednostranného, například kauzálního působení.}
}

\newglossaryentry{mouseX}
{
  name={mouseX},
  description={proměnná záskávající pozici kurzoru na plátně Processingového programu v ose X}
}

\newglossaryentry{mouseY}
{
  name={mouseY},
  description={proměnná záskávající pozici kurzoru na plátně Processingového programu v ose Y}
}

\newglossaryentry{rectMode()}
{
  name={rectMode()},
  description={funkce ovldající mód centrování obdélníku, možné parametry jsou pouze čtyři:  rectMode(CORNER) je základní chování funkce \vyraz{rect()} tedy prvními dvěma parametry umístíme nejdříve prní roh v ose x a y a poté definujeme relativní šírku a výšku odélníku (parametry pro šířku a výšku můou být i záporná čísla); rectMode(CORNERS) definuje dva rohy separátně tedy v asolutních koórdinátech na ploše; rectMode(CENTER) zavoláním funkce \vyraz{rect()} vycentruje obdélník kolem počátku x a y, pomocí dvou dalších parametrů opět funkce \vyraz{rect()} definuje šířku a výšku centrovaného obdélníku; funkce rectMode(RADIUS) definuje opět středové body x a y v prvních dvou parametrech, druhé dva parametry udávají polovinu šířky a polovinu výšky obdélníku}
}

\newglossaryentry{rect()}
{
  name={rect()},
  description={funkce pro kreslení obdélníku, přijímá čtyři parametry počátek x, počátek y, šířku a výšku. Centrování obdélníku se dá modifikovat funckí rectMode()}
}

\newglossaryentry{background()}
{
  name={background()},
  description={funkce nastavuje barvu pozadí na plátně. Standartní barva je světle šedá. Zavoláním background s definicí barvy v kulatých závorkách bude vyplněna celá plocha jednolitou barvou}
}


\newglossaryentry{void}
{
  name={void},
  description={neboli prázdná funkce, z anlgického \uv{prázdno} funkce která nevrací zpět žádný výsledek, funkce která spouští sérii zadaných příkazů uzavřenýc ve složených závorkách za svoji definicí}
}

\newglossaryentry{indenting}
{
  name={indenting},
  description={zarovnávání kódu do úhledných paragrafů, slouží k lepší orientaci programátora v kódu}
}


\newglossaryentry{syntax highlighting}
{
  name={syntax highlighting},
  description={barevné značky slouží k lepší orientaci programátora v kódu.}
}

\newglossaryentry{flow}
{
  name={flow},
  description={nebo-li plynutí, tok, je zvláštním stavem mysli popisovaným programátory, jedná se o stav kdy je člověk plně zanořen do práce a jakékoli vyrušení z tohoto stavu si vyžaduje opětovné nastolování, podle zkušených progrmátorů nastolení takového stavu obvykle trvá 10-15 minut práce s kódem.}
}

\newglossaryentry{draw()}
{
  name={draw()},
  description={draw je kreslící funkcí Processingu, veškerý kód uzavřený v této funkci bude vykreslen jednou za okénko, v závislosti na náročnosti pak standartně šedesátkrát za vteřinu, tuto kreslící funkci je nezbytná pro animovaný výstup nebo interakci s uživatelem}
}

\newglossaryentry{stroj}
{
  name={stroj},
  description={Stroj je technické zařízení, které přeměňuje jeden druh energie nebo síly v jiný - ať už kvalitativně nebo kvantitativně. Původně byly stroje jen mechanické, ale dnes se tak označují i zařízení pracující na jiných fyzikálních či technických principech - například elektrický transformátor. Strojem je v této knize téměř výhradně myšlen počítač. Počítač je programovatelný typ stroje který přijímá vstup, ukládá a zpracovává data a umožňuje výstup v požadovaném formátu.},
  plural={stroje}
}

\newglossaryentry{slovník}
{
  name={slovník},
  description={právě se zde nacházíte},
  plural={slovníky}
}


\newglossaryentry{true}
{
  name={true},
  description={pravda, neboli 1}
}

\newglossaryentry{false}
{
  name={false},
  description={nepravda, neboli 0}
}

\newglossaryentry{boolean}
{
  name={boolean},
  description={datatyp který může mít jen dva stavy \vyraz{true} a \vyraz{false}}
}

\newglossaryentry{Built with Processing}
{
	name={Built with Processing},
	description={doslova znamená: \uv{Postaveno s Processingem}. Jedná se o zvláštní komunitní frázi, která se objevuje se u projektů využívajících Processing. Fráze vyjadřuje vděk všem participantům a tvůrcům Processingu za tvorbu tohoto nástroje.}
}

\newglossaryentry{otevřený software}
{
  name={otevřený software},
  description={software s veřejně dostupným zdrojovým kódem},
  plural={otevřeného softwaru}
}

\newglossaryentry{GNU / Linux}
{
  name={GNU / Linux},
  description={Gnu Is not Unix, GNU je projekt založený Richardem Stallmanem, jedná se o operační systém a rodinu programů s otevřeným zdrojovým kódem.}
}

\newglossaryentry{GNU / GPL}
{
  name={GNU / GPL},
  description={jedna z licencí otevřeného softwaru zaručující otevřenost kódu, kterou v případě dalšího použití vyžaduje i u programů, které tento kód využívají}
}

\newglossaryentry{MIT licence}
{
  name={indenting},
  description={je licence kompatibilni s licencí \slovnik{GNU / GPL}, jedná se o speciální licenci Univerzity MIT --Massachusetts Institute of Technology},
  plural={MIT licencí}
}


\newglossaryentry{kompilace}
{
	name={kompilace},
	description={proces překladu z čitelného textu do strojového kódu}
}

\newglossaryentry{zdrojový kód}
{
	name={zdrojový kód},
	description={kód čitelný pro člověka},
	plural={zdrojové kódy}
}

\newglossaryentry{binarní kód}
{
	name={binární kód},
	description={též strojový kód, slovo vycházející z anglického slova {\em binary} dvojkové soustavy, v počítačové terminologii označuje strojový kód, tedy kód prošlý procesem \slovnik{kompilace}.},
	plural={binární kódy}
}

\newglossaryentry{Java}
{
	name={Java},
	description={Java je objektově orientovaný programovací jazyk, který vyvinula firma Sun Microsystems a představila 23. května 1995.}
}

\newglossaryentry{sketch}
{
	name={sketch},
	description={nebo-li {\em náčrt}, je koncept v prostředí Processing uchovávání jednotlivých projektů do adresářů, jedna {\em sketch} je adresář, kam Processing ukládá data uživatele (tj. především kód a externí soubory)},
	plural={sketche}
}

\newglossaryentry{CODE}
{
	name={CODE},
	description={adresář uvnitř \slovnikpl{sketch}, který obsahuje externí \slovnik{Java} \slovnik{binární kód} programy nebo knihovny třetí strany (obrázky,textové soubory, atd.)}
}

\newglossaryentry{DATA}
{
	name={DATA},
	description={adresář uvnitř \slovnikpl{sketch}, který obsahuje vaše externí soubory (obrázky,textové soubory, atd.)}
}
