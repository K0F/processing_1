% !TEX TS-program = xelatex
% !TEX encoding = UTF-8

% This is a simple template for a XeLaTeX document using the "article" class,
% with the fontspec package to easily select fonts.

 % use larger type; default would be 10pt
\documentclass[11pt,twopage]{book}
\usepackage[czech]{babel}
\usepackage{ucs}
% Unicode support for LaTeX character names (accents, European chars, etc)
\usepackage{fontenc}
\usepackage{xltxtra,fontspec,xunicode}

%\usepackage[utf8x]{inputenc}

% syntax highlight
\usepackage{listings}
\usepackage{xcolor}
\usepackage{makeidx}
\usepackage{index}     % balík pro indexování

\makeindex


% pomocné příkazy
%\def\cindex{\index*[cmnd]}        % zařazení do rejstříku příkazů
%\newcommand{\hla}[1]{\textit{#1}} % pro vysvětlení pojmu



% colors
\definecolor{javared}{rgb}{0.0,0.4,0.8} % for strings
\definecolor{javagreen}{rgb}{0.55,0.55,0.55} % comments
\definecolor{javapurple}{rgb}{0.8,0.4,0,0} % keywords
\definecolor{javadocblue}{rgb}{0.45,0.45,0.45} % javadoc


% define backgroundcolor
\definecolor{bggray}{rgb}{0.97, 0.97, 0.97}


% lsset, kód jak vypadá
% Add your keywords here,  and have this in a separate file
% and include it in your preamble
\lstset{emph={abs, acos, alpha, ambient, ambientLight, append, applyMatrix, arc, arraycopy, asin, atan, atan2, background, beginCamera, beginRecord, beginShape, bezier, bezierDetail, bezierPoint, bezierTangent, bezierVertex, binary, blend, blendColor, blue, box, brightness, camera, ceil, colorMode, concat, constrain, cos, createFont, createGraphics, createImage, createWriter, cursor, curve, curveDetail, curvePoint, curveTightness, curveVertex, day, degrees, delay, directionalLight, dist, draw, ellipse, ellipseMode, emissive, endCamera, endRecord, endShape, exit, exp, expand, fill, frameCount, filter, floor, frustum, get, green, hex, hint, hour, hue, image, imageMode, join, keyReleased, keyTyped, lerp, lerpColor, lightFalloff, lights, lightSpecular, line, link, loadBytes, loadFont, loadImage, loadPixels, loadStrings, log, loop, mag, map, match, max, millis, min, minute, modelX, modelY, modelZ, month, mouseClicked, mouseDragged, mouseMoved, mouseReleased, nf, nfc, nfp, nfs, noCursor, noFill, noise, noiseDetail, noiseSeed, noLoop, norm, normal, noSmooth, noStroke, noTint, open, openStream, ortho, param, perspective, list, beginDraw, endDraw, blend, copy, mask, set, point, pointLight, popMatrix, pow, printCamera, printMatrix, printProjection, close, flush, print, println, pushMatrix, quad, radians, random, randomSeed, rect, rectMode, red, redraw, resetMatrix, reverse, rotate, rotateX, rotateY, rotateZ, round, saturation, save, saveBytes, saveFrame, saveStrings, scale, screenX, screenY, screenZ, second, setup, shininess, shorten, sin, size, smooth, sort, specular, sphere, sphereDetail, splice, split, splitTokens, spotLight, sq, sqrt, status, str, charAt, equals, indexOf, length, substring, toLowerCase, toUpperCase, stroke, strokeCap, strokeJoin, strokeWeight, subset, tan, text, textAlign, textAscent, textDescent, textFont, textLeading, textMode, textSize, texture, textureMode, textWidth, tint, translate, triangle, trim, unbinary, unhex, unhint, updatePixels, vertex, year, keyPressed, mousePressed, frameRate, String, color}
, emphstyle={\color{javapurple}}%
}%



\lstset{
breaklines=true,
language=Java,
basicstyle=\footnotesize\ttfamily,
frame=single,
framesep=10pt,
keywordstyle=\color{javapurple},
stringstyle=\color{javared},
commentstyle=\color{javagreen},
morecomment=[s][\color{javadocblue}]{/**}{*/},
stepnumber=2,
numbersep=10pt,
tabsize=4,
showspaces=false,
showstringspaces=false,
backgroundcolor=\color{bggray},
literate = {{š}{{\vs}}1 {ě}{{\ve}}1 {č}{{\vc}}1 {ž}{{\vz}}1 {ř}{{\vr}}1 {ň}{{\vn}}1},
inputencoding=utf8,
extendedchars=false
}






% \usepackage{fontspec} % Font selection for XeLaTeX; see fontspec.pdf for documentation
% \defaultfontfeatures{Mapping=tex-text} % to support TeX conventions like ``---''

\usepackage{fontspec}
%%% HEADERS & FOOTERS
%\usepackage{fancyhdr} % This should be set AFTER setting up the page geometry
%\pagestyle{fancy} % options: empty , plain , fancy
%\renewcommand{\headrulewidth}{0pt} % customise the layout...
%\lhead{\leftmark}\chead{}\rhead{\emph{Processing 1.0}}
%\lfoot{}\cfoot{\thepage}\rfoot{}

\setmainfont{Liberation Serif} %URW Gothic L,DejaVu Sans,Georgia,Jumpcrew Cologne,Libertinage,LMSans12, Nimbus Sans L
\setromanfont{Liberation Serif}
\setsansfont{Liberation Sans}
\setmonofont{Liberation Mono}

% other LaTeX packages.....
\usepackage{geometry} % See geometry.pdf to learn the layout options. There are lots.
\geometry{a4paper} % or letterpaper (US) or a5paper or....
%\usepackage[parfill]{parskip} % Activate to begin paragraphs with an empty line rather than an indent


\usepackage[dvips]{hyperref}
\usepackage{graphicx} % support the \includegraphics command and options
\usepackage{verbatim}

\title{Processing}
\author{Kryštof Pešek}
\date{} % Activate to display a given date or no date (if empty),
         % otherwise the current date is printed 



%bibliografie
\bibliography{ProcessingBib}


\begin{document}

\tableofcontents


\chapter{Preludium}


\section{Úvodní slovo}

Vážený čtenáři,

tato kniha by Vás měla povzbudit v nesnadné úloze osvojení si vašeho prvního programovacího jazyka. Není-li tento jazyk první, prosím Vás o shovívavost v podrobnostech, do kterých v úvodu této knihy zabíhám. Šíře znalostí, které se mohou vyskytovat u potenciálních čtenářů tohoto průvodce je pro autora první velikou neznámou.

V obou případech Vás tedy prosím o shovívavost ve způsobu popisu programovacího jazyka a prostředí Processing. Sám jako samouk nemohu zaručit absolutní, stoprocentní a všeobecnou platnost všech tvrzení. Tímto Vás tedy žádám o věcné podněty pro pozdější doplnění nebo přeformulování jednotlivých trzení.

Má snaha provést začínající i středně pokročilé uživatele jazykem bude vždy nezbytně nedostatečná, berte ji prosím spíše za průvodce nesnadnými začátky.

Čtete-li knihu z jiných důvodů, než z důvodu učení se programovacímu jazyků, pokusím se text knihy proložit poznatky nabytými moji několikaletou zkušeností sdílení života se stroji.

Dostala-li se vám tato kniha do rukou jinou cestou nebo dokonce náhodou nebo omylem, tedy věru netuším, co v ni dále najdete, a proto bych i Vás rád pobídl alespoň k začtení se do světa skriptů a kódů. 

\section{Jak číst tuto knihu?}

\subsection{Návod ke čtení knihy}

Text je řazený do jednotlivých kapitol a podkapitol jejichž pořadí by mělo odpovídat procesu učení se novému programovacímu jazyku. Pořadí a obsah jednolivých kapitol vychází ze zkušenosti s vlastním učením se, a zároveň koresponduje s podobnými ofic-iálními průvodci psanými tvůrci programovacího jazyka Processing. Jedná se o ověřený postup, který by měl co možná nejsrozumitelnější formou podat základy programovacího jazyka Processing.

Následující text bude podléhat určitým zákonitostem. Holý text (který nyní čtete) je člověku srozumitelný popis. Text v šedém rámování bude značit samotný strojový kód kerý je strozumitelný Processingu. Výrazy psané {\em kuzívou} budou vždy znamenat výrazy, jejichž definici můžete nalézt ve jmenném.

, tj. na konci knihy. Takové výrazy, budou mít vždy jasnou definici.



Pravidla:

\begin{enumerate}
\item
Klasický text, rozdělený do kapitol je pouhý popis. Tento text Vás bude provázet celou procedurou.


\end{enumerate}


\section{Jazyk}

\subsection{Definice jazyka}

\begin{enumerate}
\item
K tomu, abychom sdělili informaci, používáme jazyk.


\item
Jazyk musíme umět přisbůsobit tomu, aby informoval, sdělil jistou skutečnost - myšlenku sdělovanému subjeku.

\item
Jazyk má nutně několik úrovní, zdaleka ne všechny jsme schopni reflektovat.

\item
Úrovně, které jsme schopni reflektovat, jen částečně umíme logicky popsat.

\item
Logicky popsaným jazykem a vnitřně uceleným systémem jsme schopni vysvětlit pouhý fragment skutečnosti.

\item
V případě programovacího jazyka si je nutné uvědomit, že programujeme (inforujeme) stroj. Dnešní stroje ve fyzi jen zanedbatelně velikou skutečnost fyzicky nepřesahující dimenze součástí stroje\footnote{nejvíce pak procesoru paměti, skutečnost na pevném disku a grafických procesorů}, ale tím zpětně i sebe sama a druhé lidi\footnote{potažmo jakkoli širší skutečnost (zvířata, rostliny, cokoli o čem ani dále netušíme)}, symbolicky a nepřímo popisujeme zkratku skutečnosti nepoměřitelně daleko větší.
\end{enumerate}


\subsection{Jednoduchost jazyka}

Snaha po spřístupnění programování širší veřejnosti dala již na konci dvacátého století vzniknout rodině jazyků, které jsou patřičně zjednodušeny tak, aby je mohli obsluhovat i neodborníci.

Výchozím bodem pro zjednodušení programování je odpověd na situaci, kdy programovací jazyky vytvářeli především lidé se zvláštním nadáním pro ryze technické uvažování. Technické uvažování je speciální schopnost, pro kterou nemá každý člověk správné predispozice. Současně, programování v dnes znamená určitý stupeň širší svobody při komunikaci se strojem. K míře svobody, která má své silné kritiky\footnote{včetně mne samotného}, se nyní nechci vyjadřovat, ale zjednodušeně z pohledu pouhého uživatele který používá daný nástroj, schopnost programovat činí z uživatele již potencionálního strůjce vlastních nástrojů.

Hovořím-li o stroji, mám dnes na mysli spotřební počítač. Termín {\em stroj} používám záměrně pro zdůraznění jisté formy strojového přemýšlení v historickém kontextu.\footnote{V současné době se setkáváme nejčastěji s počítači založenými na mikroprocessorech výrobců Intel, AMD a v poslední době rostoucí firmou zabývající se velmi drobnými processory Atmel. Architektury procesorů dnes nejšastěji rozdělujeme na 32 a 64-bitové processory. Nárokům trhu na větší výkon stroje se dnes vychází vstříc množením jader v jednom stroji.}

Nástroje jsou pak programy zkonstrouvané pro jistou činnost. Obvykle je nástroj vyvíjen za jedním účelem, který plní uživatelsky co nejpřivětivější cestou. Tato cesta je pro uživatele snadno schůdná a nabízí mu standartní škálu dovedností nástroje. Aniž bychom si toto přiliž uvědomovali, současná vizuální kultura je ovlivněna těmito nástroji daleko více než je na první pohled patrné. A technické možnosti jsou současným tržním hladem pro inovaci konfiskovány a proměnovány ve zboží. V této situaci je důležité znát nástroje i jejich vznik pro reflexi nebo kritiku v širších souvislostech.

Tato kniha je spíše než-li jednomu nástroji věnována programu pro tvorbu takových nástrojů. Jak již vyplývá z této definice použití Processingu není limitováno jen úhlem pohledu autora tohoto textu. Návod by se měl stát spíše pobídkou k co nejrozmanitější tvorbě vlastních nástrojů, sloužících opět pokudmožno k co možná neširší škále možných účelů.

Processing vychází z koncepce snadného přístupu k programování. Za běžných okolností by se vnímavý člověk měl být schopen naučit jednoduché struktuře programu a schopnosti vytvořit vlastní program v průběhu několika dní. Na druhou stranu, právě predispozice našeho uvažování jsou natolik rozmanité, že takovou prognózu nelze brát jinak než-li jen za orientační.


Programovací jazyk processing vychází z laboratoří MIT, kde se celé oddělení věnuje právě koncepcím zjednodušování programovacích jazyků .... historie small talk atd. zde


\subsection{Dokonalost jazyka}

Co je to dokonalý jazyk? Nejprve je zapotřebí říci, že absolutně dokonalý jazyk neexistuje. Jazyk si můžeme představit jako systém vzájemných vztahů, který je schopen popsat jednotlivé symboly nebo objekty. Symboly můžeme nazvat předměty, tyto předměty dále mají své vlastní hodnoty a vlastnosti, jazyk kromě definic takových vlastností operuje a popisuje jednotlivé jevy a vztahy mezi těmito předměty. Jednodušší popis jazyka je v pojetí výpočetní techniky určitý ucelený systém schopný popsat rozmanité problémy řešitelné strojem.

Co činí předem jakýkoli jazyk absolutně nedokonalým je nejprve fakt, že jakýmkoli jazykem nedokážeme vyjádřit původ jazyka, tj. jeho strůjce; člověka. Jazyk použitý pro instuktáž stroje je vnitřně konzistentní a funguje logicky hermeticky, tj. nepřipouští jiný než jeden výklad konkrétního textu. Pojetí dokonalosti jazyka ve smyslu vnitřní logické konzistence je naprostou nezbytností v pojetí interpretace strojem, na druhou stranu téměř nepřekonatelnou překážkou v případě abstraktnějšíh úvah o programování jako takovém.

Použiji zde pro názornost rozdíl programovacího jazyka s jazykem českým. Český jazyk, stejně tak jako jakýkoli mluvený nebo psaný jazyk, je jazykem organickým ustáleným po staletí užívání. Jazyk jak ho známe slouží ke komunikaci mezi lidmi, lze tedy použít například pro popis krajiny. Přestože k dokonalému popisu krajiny ztěží kdy můžeme dojít, těmito slovy které se opírají o určitou sdílenou zkušenost, lze poměrně dobře přiblížit určitý obraz věcí.

Pokusili bychom se pro popis krajiny použít jazyk programovací, dostaneme se velmi rychle do nesnází. Programovací jazyk není jazykem určeným primárně pro předání informací mezi lidmi, ale pro komunikaci člověka se strojem. Jeho vnitřní logická konzistence, tvrdá logická struktura, která nedovoluje v jeden okamžik jinou než jednu interpretaci, je jeho velikou předností při definici exaktních parametrů. Podobnost s řečí spočívá ve vazbě slov, které reprezentují jednotlivé hodnoty a operace. Hlavní odlišnost je v jeho syntetickém původu, jedná se o jazyk umělý a účelu ke kterému byl zkonstruován. Programovací jazyk je přednostně zkonstuovaný pro definici známého a pochopeného. V případě neznámých nebo nepoznaných veličin, je programovací jazyk víceméně k ničemu. 

Chceme-li komunikovat se strojem musíme tedy svůj způsob vyjadřování přizpůsobit logicky dokonalému jazyku - vnitřní logice fungování stroje. Počítač není navržen k tomu aby něčemu rozuměl, počítač je navržen k řešení jasně definovaných otázek. Tato kniha se pokusí srozumitelnou formou popstat jeden z možných způsobů jak si osvojit takový jazyk a potažmo způsob uvažování, který vede k jasné definici problému. Hovoříme-li o programování, máme na mysli proces tvorby jisté logické struktury. Osvojení si programování spočívá ve schopnosti definovat problém nebo jasně formulovat otázku, tak aby stroj na ni mohl odpovědět.

Vtip celé věci spočívá v tom, že ovládneme-li formálně jazyk určený stroji, můžeme prostřednictvím tohoto stroje hovořit i ke člověku, tj. popisovat pocity z rozkvetlých luk.

\subsection{Volba jazyka}


Ve výpočetní technice se nachází celá škála programovacích jazyků i prostředí. Tyto jazyky mají svoji genezi a byli historicky vyvíjeny především počítačovými odborníky. Jejich dokanalost lze těžko ocenit z vnějšího pohledu, a to právě z důvodu jejich konstrukce, která odpovídá a částečně podléhá určitým účelům, ke kterým byli tyto jazyky původně navženy. Celistvý pohled na vývoj programovacích jazyků zde není možné obsáhnout. Základní rozdělení programovacích jazuků dle historického vývoje můžeme dnes označit na dvě skupiny, jazyky procedurální a objektově orientované.

Processing se svojí stavbou na základech Javy řadí k objektově orientovaným jazykům. Jeho společnými příbuznými jsou kromě {\em Javy} jazyky jako {\em C++, C\#,} nebo {\em VisualBasic}. Toto rozdělení pojednává o koncepci jazyka a jeho základních struktur. Ke kterým se dostaneme později. 

V posledních přibližně dvaceti letech se mezi programovacími jazyky postupně objevuje tendence po větší srozumitelnosti a potažmo zjednodušení programovanání jako takového. Programování v této koncepci zjednodušování již není jen jazykem odborníků, ale je demokraticky přístupný širší veřejnosti z rozmanitých - prioritně netechnických oborů.

Tato tendence postupně dala vzniknout celé rodině programovacích jazyků
\footnote{vyjmenovat jazyky zde + historie?}
které mají ve snaze nablížit potenciál výpočetní techniky blíže k výtvarné tvorbě. V technických kruzích je již sama disciplína psaní programů považována za tvůrčí. V pojetí výtvarného umění dnes převažuje pohled na programování jako na rigidní a notně limitovanou činnost.

Jazyk, který je nutně limitovaný svojí vnitřní dokonalostí (dokonavostí?) ovšem nemusí nutně limitovat jeho uživatele ve sdělení. Uživatel se ovšem musí do určité míry, naučením se základních struktur, přizpůsobit stroji k uskutečnění zdárné komunikace.

[Důvody pro processing]

\section{Tvorba softwaru}

\subsection{Otevřenost softwaru}

Jazyk nazvaný Processing je jedním z jazyků, který byl vytvořen v diskurzu zjednodnušování programování. Jako každý jiný programovací jazyk je i Processing navržen pro jisté účely. V případě tohoto programovacího jazyka se nejvíce jedná o důraz na rychý vývoj a zjednodušené nakládání s obrazem i prostorem. Z více technického pohledu pak processing vyniká otevřeností zdrojového kódu a důraz na multiplatformnost.

Jazyk i programovací prostředí Processing je v současnosti tzv. otevřený software, což znamená, že i samotný zdrojový kód je volně k dispozici a šiřitelný pod MIT licencí [dpolnit licence GPL 3.0 .. atd]. Z pohledu uživatele je velmi příjemné, že samotný software je k dostání zdarma na stránkách projektu. Za jeho užívání není nutné platit žádné poplatky, a to i v přpadě komerčních užití.

Z pohledu vývojáře znamená oteřenost zdrojového kódu zásadní věc, jednoduše pro dosažitelnost celého zdroje, který se dá následně například implementovat do různých prostředí, nebo kvůli možnosti rozšířit jazyk o vlastní knihovnu. Otevřenost kódu udržuje počet participantů a de facto celý projekt v dlouhodobém horizontu naživu.

Processing na první pohled není ničím zvláštní programovací jazyk. V podstatě by se dalo říci, že se jedná pouze o rozsáhlou knihovnu pro jazyk Java. To co Processing řadí mezi oblíbené softwary pro tvorbu dnes je nejvíce přívětivá komunita uživatelů s velmi odlišnými stupni znalostí a úhly pohledu. Zvláštní důraz je v komunitě kladen na poskytnutí co největší podpory právě začínajícím uživatelům. Tomu odpovídá i počet rozmanitých průvodců a rozsáhlá, velmi dobře stučně napsaná dokumentace ke každému z příkazů v Processingu.

K otevřenosti v kódu v neposlední řadě přistupuje i celá řada velmi zkušených tvůrců. Tím se uživatel na jakémkoli stupni znalostí může kdykoli naučit nové postupy nebo může svobodně recyklovat algoritmy druhých uživatelů. Processing a čím dál více jeho uživatelů ctí filosofii otevřeného softwaru která (mimo jiné) hlásá: {\em Vědomosti nesmí být privatizovány!}

\subsection{Procesualita, živý program}

Hlavní doménou Processingu je schopnost vytvářet "živé" programy, tj. programy běžící v reálném čase. Již jméno programovacího jazyka Processing napovídá akcent jisté procesuality, v čase se odvíjející události.

Obraz vytvořený tímto způsobem, na první pohled zaměnitelný s videem, nemusí například podléhat časové omezenosti, nebo může určitým způsobem reagovat na své okolí. Generovaný obraz může být například takto vtažen do kontextu, ve kterém se nachází. Možností jak nakládat s těmito specifickými možnostmi je nepřeberné množství.

[rozvést více]

\subsection{Empirický přístup k programování}

	I když zásadní odlišnost s jinými programovacími jazyky bychom hledali ztěží, processing proslul zejména pro snadnost použití. Kompilovat program není otázkou nastavování kompileru a veškerých jaho parametrů, program jednoduše po stisku tlačítka {\em RUN} běží (je-li správně napsán). Samozřejmě tento redukcionistický přístup má své nevýhody, speciálně při rozsáhlejších projektech tato jednoduchost může dokonce omezovat. Processing ovšem jako svobodný software lze naimplementovat do řady jiných prostředí, a potřebujete-li si kompilační proces nastavit sami, nic Vám například nebrání použít Processing jen jako knihovnu do {\em Javy}.

Tato základní jednoduchost na druhou stranu nezdržuje uživatele od myšlenkového toku psaní programu. Častou kontrolou výsledku kódu uživatel může lépe interagovat se samotným tvarem programu.

Nazývám tento způsob programování empirický, tedy přístup, kdy podle zkušenosti s běžícím programem je dotvářen i samotný zdrojový kód. Mnoho technicky zaměřených lidí by zřejmě mohlo tento postup kritizovat pro přílišnou reduktivnost a amatérský přístup k programování. Zde bych oponoval faktem, že motivace lidí vytvářející například instalaci do galerie nezajímá příliž dokonalost (když už tedy vůbec) programu. Program je v pojetí tvůrců jen prostředníkem pro další sdělení, a jestliže toto sdělení předá je to dobrý program.

Proces poznávání struktur jazyka při tvorbě obrazu prostřednictvím processingu bych přirovnal k postupu od začátečnické malby na tkaninu k postupnému tkání gobelínu. Zde je nutno podotknout, že ne každý tvůrce využívající Processing chce tkát gobelín a najde-li nástroj vhodný "jen" pro "malbu" na plátno, je to dobrý nástroj.


[????]

\chapter{Processing jako prostředí}

\section{Základní prostředí}

Processing představuje ucelené programovací prostředí tzv. IDE\footnote{Integrated Development Enviroment}. Jedná se o kompletní prostředí určené především k rapidnímu vývoji aplikací. Samotný program je jedna otevřeným softwarem\footnote{tj. s otevřeným veřejně dostupným zdrojovým kódem} a zdarma ke stažení pro všechny majoritní platformy tedy pro GNU / Linux, Mac OS x i pro Windows na stránkách projektu processing.org. Celý jazyk i samotné IDE vychází pod licencí GNU/GPL 2 jedná se tedy o svobodný software. Díky této licenci můžete jakýkoli výstup z tohoto softwaru můžete publikovat, použit pro jakékoli účely v četně např. komereční aplikace.


Základní rozhraní tvoří textový editor s několika nezbytnými funkcemi. Patrně nejdůležitější je v liště tlačítko (1) tzv. RUN. Které kompiluje a spouští program aktuálně rozpracovaný v textovém editoru. (2) Tlačítko STOP naopak program zastaví. V případě chyby v programu lze též využít jako vynucené zavření programu. Zbylé tlačítka slouží k diskovým operacím. \\


\begin{center}
\includegraphics[scale = 1]{imgs/buttons.png}
\end{center}


\newpage
\subsection{Základní diskové operace}

Ty operace jež něco zapisují či načítají z disku. Tyto opera zahrnují veškerou manipulaci se sketchí, její vytvoření, uložení, načtení či export do webového formátu či kompilaci do spustitelné aplikace.

Processingová sketch nemusí být nutně uložena aby jste ji mohli spustit. Nové úpravy, které v textovém poli napíšete budou dočasně uloženy v závislosti na vašem operačním systému jinde. Tímto způsobem můžete experimentovat s kódem aniž by jste přišli o předchozí verzi.

Mezi základní operace patří vytvoření nové sketche (3), otevření předešlé (4), uložení (5) a export (6).

Tlačítka mají dále speciální vlastnosti s podržením tlačítka
\index{SHIFT}
tlačítko s novou sketchí otevře také nové okno editoru. V kombinaci s načtením předchozí sketche (4) otevřete též sketch v novém editoru aniž by jste ztratili předchozí okno. U uložení se modifikace projevuje funkcí uložit jako. U exportu stisknutím klávesy {\em SHIFT} přepínáte mezi exportem samostatné aplikace nebo tzv. appletu, aplikace schopné běžet v prohlížeči nebo obecně na webových stránkách.

\subsection{Editor}

Editor je textové pole a je vaším hlavním komunikačním nástrojem s Processingem. Veškeré informace které zadáte do tohoto pole budou interpretovány samotným Processingem. Textový editor není nijak dokonalý, pro začátek si s ním ovšem vystačíme. Formát který tento editor produkuje je ve své podstatě textový soubor s příponou {\em *.pde}.

Textový editor má několik funkcí, které vám mohou usnadnit práci. Odlišuje například mezi příkazy Processingu odlišnými barvami, tato zdánlivá banalita vám speciálně v začátcích nebo v případě objemnějších kódů může výrazně zpřehlednit organizaci programu. Jedná se o programátorskou konvenci obecně v anglickém jazyce nazývanou {\em syntax highlighting}.

Další konvencí se kterou se často setkáte je tzv. {\em indenting}, zarovnávání kódu do úhledných paragrafů. V Processingu si ze začátku vystačíte se standartím zarovnáním, posléze lze zarovnávání aktivovat stiskem kombinace kláves {\em CONTROL (APPLE) + T}.\\


\begin{center}
\includegraphics[scale = 0.75]{imgs/editor.png}
\end{center}

Horní lišta editoru označuje záložky, kliknutím na šipku v na pravé straně lišty se vám zobrazí možnosti operací se záložkami. Pro začátek s nimi nebudeme operovat. V případě delšího kódu záložky slouží k lepší organizaci struktury programu. V podstatě se dá říci, že každá záložka odpovídá jednomu souboru ve sketchi. Co je to sketch popíšu níže.

Jak můžete vidět na obrázku editor operuje i českými znaky, ačkoli bych jejich používání z důvodů internacionalizace neradil používat. V komentářích tj. textem uvozeným dvěma dopřednými lomítky {\em //}, tedy text, který kompiler ignoruje, jsou diakritická znaménka přípustná.

Naopak v případě názvu proměnných, nebo jakýchkoli funkčních definic, je použití diakritiky v devadesáti procesntech nefunkční.\footnote{standartní jazykové kódování, ve kterém Processing operuje je {\em UTF-8}}

\subsection{Sketch}

Sketch, neboli náčrt, označuje samostatný projekt Processingu. Dělením projektů na sketche jsou uspořádány jednotlivé projekty processingu. Sketch je ve své podstatě složka obsahující alespoň stejně jeden stejně nazvaný soubor s příponou {\em *.pde}. Sktech při vytvoření nevyžaduje název, je jí přidělen pouze aktuální datový kód zaznamenávající datum vytvoření. Tímto decentním způsobem processing toleruje například nejasnost záměru autora při vytváření nového díla, jeho název či pracovní název lze tímto způsobem přiřadit až později, po nabytí jasnějších obrysů.

Program si vystačí pouze se samotným zdrojovým kódem tj. souborem (soubory) {\em *.pde}. Ovšem v případě operujeme-li s externími daty stojícími mimo zdrojový kód, jakékoli jiné soubory, můžeme využít dvě možnosti. První možnost je soubor, se kterým potřebujeme operovat tažením myši přesunout na textové pole Processing IDE. Touto operací bude soubor zapsán do naší sketche automaticky. Druhá možnost je operaci provést manuálně, stačí vytvořit v adresáři sketche adresář s názvem {\em DATA}. Processing tuto složku automaticky rozpozná a soubory v této složce budou dobře dostupné pro pozdější operace.

Ke standartním adresářum, které se často objevují ve struktuře skteche, si stačí zapamatovat jen dva již zmíněný adresář nazvaný {\em DATA}. Tento adresář je využíván při práci s jakýmikoliv vnějšími daty jako jsou např. obrázky, zvuky, videa, vektorové grafiky nebo například textové soubory. Druhý adresář nazvaný {\em CODE} obsahuje externí zdrojový kód, popřípadě kód kompilovaný v podobě knihovny, mající nejčastěji přípony {\em *.java, *.class, *.jre}. Tento adresář se nalézá v systémové cestě daného projektu a umístíme-li zde soubory budou též dobře dostupné z daného projektu. Adresář {\em CODE} nás zatím nemusí příliž zajímat, později se jeho prostřednictvím můžeme pokusit rozšiřovat funkce Processingu.

\subsection{Sktechbook a uspořádání}

Místo na disku, které Processing využívá k uskladnění sktechí se nazývá povšechně Sketchbook a je v podstatě pouze adresář obsahující jednotlivé projekty. Koncepce sketchbooku spočívá v uspořádání jednotlivých projektů do organizované formy a ačkoli nevnáší do jednotlivých projektů sám o sobě řád může napomoci k vytvoření řádu vlastního. Ze své zkušenosti mohu říci, že organizace jednotlivých projektů do jakékoli struktury má opravdu smysl. Mnou preferovaná struktura obsahuje jednotlivé roky následně ještě dělené do měsíců, ale možnosti organizace záleží čistě na uživatelových preferencích. Tímto chci spíše ilustrovat možnosti uspořádání než-li nezbytnost, ale přítomnost vašeho systému od počátku vřele doporučuji.

Ve sketchbooku se dále nachází jeden speciální adresář nazvaný {\em libraries}. Adresář v sobě uchovává externí rozšíření processingu o komunitní knihovny. Standartně Processing již své základní knihovny nese v sobě, tj. jsou standartně přidány do samotného processingu. V tomto adresáři se nacházejí knihovny které budete moci použít později. Na stránkách projektu {\em http://processing.org} můžete nalézt velké množství komunitních knihoven s dobrou dokumentací, volně ke stažení. Veškeré tyto knihovny tedy processing potřeuje nalézt v tomto adresáři.

\subsection{Knihovny}

Ve skecthbook u dále nachazejí (od verze ...) veškeré rozšíření za pomocí takzavných knihoven neboli libraries. Ty jsou umístěny v adresáři sketchbooku ve složce {\em libraries}. Knihovny jsou silným rozšířením celého jazyka a pokrývají mnoho možností s nakládaáním s externími daty, zajištují komunikaci zařízeními, práci s videem, zvukem, nebo s trojdimenzionálními daty či vektorovým obrazem.

Zmínka o knihovnách bývá často ve standartních průvodcích Processingem až jako pokročilá lekce. Myslím si že je dobré mít přehled o knihovnách okamžitě v úvodu. Knihovnami se dá dobře ilustrovat k čemu lidé převážně Processing využívají a jakým způsobem je rozšiřován. To nutně neznamená že jazyk musíte používat právě k těmto účelům ale je dobré mít od začátku přehled o kontextu.

Ve zkratce jak již jsem zmínil knihovny se dělí na dvě základní skupiny. Jednu tvoří interní knihovny processingu, které přichází již naistalované se softwarem. Tyto knihovny se dají nalézt pod tlačítkem Sketch > Import Library.
K základní výbavě knihoven patří tyto:

\begin{itemize}
\item
Video

Základní rozhraní mezi Apple Qucktime a Processingem. U platformy Linux, musíme využít alternativy.

\item
Network
Zajišťuje základní síťovou + internetovou komunikaci. 

\item
Serial
Podpora pro komunikaci se sériovými porty, externí zařízeni typu (RS-232).

\item
PDF Export
Knihovna pro export do PDF.

\item
OpenGL
Techologie pro podoru java implementace akcelerované grafiky JOGL.

\item
Minim
Využívá JavaSound API ke snadné obsluze zvukuvého výstupu.

\item
DXF Export
Knihovna pro exportování vektorových dat ve formátu DXF (Autocad, 3D).

\item
Arduino
Knihovna určená pro komunikaci s Arduinem.

\item
Netscape.JavaScript
Nese metody pro komunikaci s Javascriptem a processingovým appletem ve své webové podobě.

\item
Candy SVG Import
Knihovna pro načítání vektorové grafiky ve formátu SVG (Inkscape, Adobe Illustrator)

\item
XML Import
Podpora načítání XML tabulek.
\end{itemize}

Neříkají-li vám tyto zkratky nic, nevadí, jedná se většinou o datové standarty a typy zařízení se kterými můžete s pomocí knihoven pracovat. Každá z těchto knihoven má svoji dobrou dokumentaci na stránkách projektu, nebo přímo v tzv. {\em Examples}, příkladech pro každou knihovnu, lze tímto způsobem zobrazit základní nápovědu. 



\newpage
\section{Základní pravidla a zvyklosti, Hello world!}

Jako každý jazyk i Proccessing má svá pravidla. Můžeme je přirovnat ke gramatice, v počítačových programech se gramatika nazývá {\em syntax}.

Pro ukázku srovnání takových pravidel se často používá ten nejzákladnější program, tzv. {/em Hello world!}\footnote{překlad: Ahoj světe!}.

V jazyku Processing je takový program možné napsat pouze na jeden řádek:

\begin{lstlisting}
println("Hello world!");
\end{lstlisting}



\chapter{Processing jako jazyk}

\section{Orientace v prostoru}

Plocha programu by sa dala přirovnat k listu papíru, popřípadě k malířskému plátnu. Ve standartním dvoudimenzionálním módu má dva základní paramery {\em X} a {\em Y}, ty označují souřadnice, ve kterých se veškeré kreslící operace pohybují. Důležité je zmínit, že oproti jisté konvenci v matematických grafech levý horní roh nese hodnotu {\em X = 0}, {\em Y = 0}. Směrem dolů hodnota {\em Y} přibývá stejně tak jaho hodnota {\em X} přibývá směrem doprava.\\

\begin{center}
\includegraphics[scale = 1]{imgs/grid2d.png}
\end{center}

Tato konvence je převzata ze standartu počítačové grafiky, kdy první pixel v levém horním rohu nese sořadnicovou hodnotu právě {\em X = 0}, {\em Y = 0}. Obrácená osa Y se může ze začátku jevit matoucí. Důvody pro zdánlivé převrácení os pochopíme později například právě při operacích se samotnými obrazovými body, pixely, které jsou standartně uspořádány od levého horního rohu doprava a níže.

Plochu je možné představit si jako prázdný prostor, na kterém je možno zobrazovat grafiku. Tvary nebo například text se zobrazují právě prostřednictvím zdrojového kódu tj. instrukcemi psanými v editoru.

Na první pohled by se mohlo zdát, že například při zobrazení obdélníku, příkaz:

\begin{lstlisting}
rect(x,y,šířka,výska);
\end{lstlisting}

jehož výsledkem je kresba pouhého obdélníku je zbytečně komplikovaný, oproti jiným zobrazovacím metodám. Důležité je si zde uvědomit koncepci processingu, který tímto zobrazením primitivních objektů sestavuje celý obraz. To co se zpočátku může zdát jako nadbytečná práce, tj. psaním koordinátů každeho ze zobrazovaných objektů se posléze ukáže jako sofistikovaný a velmi ulehčující způsob přemýšlení o obraze. Veškeré parametry oddělené čárkou v kulatých závorkách příkazu {\em rect} náleží samotným vstupním hodnotám. Příkaz {\em rect} očekává čtyři parametry. Parametry mohou být celá čísla nebo čísla s desetinnou čárkou. Pro názornost, zadáme-li do parametrů ''natvrdo'' hodnoty:

\begin{lstlisting}
rect(8,6,7,5);
\end{lstlisting}

Výsledné zobrazení na naší pomyslné ploše bude následující:

\begin{center}
\includegraphics[scale = 1]{imgs/gridRect2d.png}
\end{center}


Zamyslíme-li se například o tom, že bychom v jakémkoli jiném nástroji chtěli zobrazit vícero, npaříklad tisíc obdélníků, na první pohled jednodušší GUI \footnote{zkratka pro "Guided User Interface" klasické grafické editory jako Gimp nebo Photoshop} grafické editory nám nedovolí tuto operaci učinit jinak, než že musíme všech tisíc obdélníků nakreslit. V případě processingu nám bude stačit vytvořit rutinu pro kreslení libovolného počtu obdélníků a pak tuto rutinu spustit.

Zde se již dostáváme k samotnému jádru Processingového přemýšlení. Programování obecně dokáže velmi ulehčit operace jejichž pravidelnost dokážeme popsat. Veškeré umění psaní programu tedy spočívá v definicích těchto chování a redukci složitýc jevů na jednoduché rovnice.

Celé řemeslné umění psaní kóduv podstatě záleží na eleganci v zápisu složitějších vztahů mezi různými parametry. Dovednosti se člověk učí postupně, osvojení si gramatické korenktnosti a logické posloupnosti se mohou zpočátku jevit zbytečně komplikované, ovšem i po ovládnutí pouhých pár jednoduchých pravidel lze Processing využít kreativním způsobem.

V další části se budeme zabývat stavbou programu. Pod odpudivým názvem se skrývají právě tato jednoduchá pravidla, která by měla tvořit základ dalším experimentům.   

\newpage
\section{Stavba programu}

\subsection{Logika programování}
Pro začátek je dobré si představit program jako sadu instrukcí. Každá instrukce má svůj význam a své místo. Instrukce se píší v programovacím jazyku a jejich interpretace je vždy pro stroj jednoznačná. Stroje podle svého návrhu nedělají nic kromě toho co mají takovým způsobem instruováno.

Veškeré programy které používáte, dokonce i programy, které nevíte že používáte byly někdy naprogramovány lidmi pomocí programovacích jazyků. Programovací jazyk je pouze, plánovací forma zápisu programu. K tomu, aby byl program spuštěn, musí být v případě jazyku Processing převeden do strojového kódu. Strojový kód se liší od toho zdrojového, našeho čitelného plánovacího jazyka, tím že není čitelný pro člověka, je čitelný pro stroj.

Počítač je "rychlý blbec". Umí rychle vykonávat sled banálních operací. Programovací jazyk slouží k vytvoření smyslu ve sledu banálních operací. Procesu převodu ze zdrojového kódu do strojového se nazývá kompilace.

O kompilaci toho nemusíme naštěstí vědět mnoho, vše bylo již tvůrci Processingu shrnuto pod tlačítko spustit.

\subsection{Základní datatypy}

Ke stavboě programu potřebujeme stavební materiál. Pro základní pochopení fungování programu je nezbytné nejdříve pochopit základní datatypy. Datatyp si lze obecně představit jako obálku na informaci.

Processing rozlišuje mezi jednotlivými datatypy. Informace, které se nacházejí v paměti se musí nacházet pod správným datatypem, tak aby program věděl jak s nimi operovat.

Pro datatyp je možné představit si různé praralely, má oblíbená je podobnost s obálkami nebo nádobami. Různé datatypy si můžeme představit jako tvary nádob. Do různých nádob, lišících se tvarem, se změstná různý obsah. Typy obsahů se dají názorně představit na rozdílu mezi textovou informací a informací číselnou.

Processing pořebuje nejdříve vědět zda nádoba obsahuje text nebo číslo, aby mohl s touto nádobou operovat. Například, nelze provést matematickou operaci mezi dvěmi slovy. Sčítat, odčítat, dělit či násobit se dají pouze čísla.


\subsection{Seznam základních datatypů}
Datatypů je jen několik, dají se vcelku snadno zapamatovat:

\begin{lstlisting}
int mojeCeleCislo = 1;

float mojeCisloSDesetinnouCarkou = 1.33;

boolean mojePravdaCiNepravda = true;

String mujSlovniRetezec = "Ahoj svete!";

char mujJednotlivyZnak = 'A';

color mojeBarva = color(255,127,1);
\end{lstlisting}

Takovému zápisu bude již Processing rozumět. Vepsáním těchto řádků do processingového editoru již dochází k alokaci vašich informací ve správných datatypech v paměti. Vysvětleme si nyní několik nejasností.

První slovo na každém řádku označuje odlišný datatyp (náš tvar nádoby), následuje jméno proměnné (název pro naší nádobu), poté následuje rovnítko které již přiřazuje obsah - hodnotu do proměnné (nádoba je naplněna).

Každý příkaz je ukončen středníkem {\em ;} .

Podivné názvy jako například {\em mojeCeleCislo} tzv. proměnných, jen ukazují, že název pro svoji proměnnou můžete zvolit téměř podle libosti. Co do vyjímek v názech proměnných. Proměnná nesmí mít v názvu mezeru tedy prázdný znak. Podle zvyklostí by dále proměnná měla začínat malým písmenem a potřebujete-li nutně pro název použít více-slový název použijte velké písmeno namísto mezery.

Proměnná nesmí být číslo, ale slovní název. Programátorské konvence při pojmneovávání proměnných jsou ryze praktické. Proměnná by měla mít co možná nejkratší a nejvýstižnější název.  Toto je spíše dobré doporučení než-li pravidlo. Správným pojmenováním proměnných docílíte větší přehlednosti v kódu kratší délkou si ušetříte zbytečné psaní.

\subsection{Tisk do konzole}

Tento jednoduchý program, zatím jen definoval pár proměnných do správných datových typů zatím nedělá nic zajímavého. Celá alokace probíhá uvnitř programu. Processing dělá většinu svých operací skrytě. Program kreslí nebo jinak interaguje s uživatelem jen je-li o to požádán.

Nejjednodušší výstup z programu je tisk do tzv. konzole. Konzoli jsme si již krátce uvedli v kapitole {\em Základní prosředí}. V processingu se konzole nachází pod textovým editorem. Toto černé pole má pouze textový výstup a slouží k odlazování programu.

Tiskem do konzole si nyní můžeme například zkontrolovat obsah našich proměnných. To můžeme provést následujícími dvěma způsoby:


\begin{lstlisting}
print(mojeCeleCislo);

println(mujSlovniRetezec);
\end{lstlisting}

Oba příkazy tisknou obsah našich proměnných. Jediný rozdíl mezi příkazy je ten, že druhý příkaz končí znakem:
 
\begin{lstlisting}
"\n"
\end{lstlisting}

jedná se o tzv. {\em newline character}, přidáním speciálního charakteru příkaz tiskne pokaždé na nový řádek. Není nutné si pamatovat tento speciální znak, postačí když si zapamatujeme přikaz {\em println( cokoli );}

Spustíme-li program nyní, v konzoli se nám ukáže výstup z našeho programu:

\begin{lstlisting}
1ahoj světe!
\end{lstlisting}

Zde je názorně vidět, že tisk pomocí pouhého příkazu {\em print} nepoložil další příkaz na nový řádek, a tedy vytiskl {\em 1ahoj} dohromady.

Tisk do konzole se hojně používá při lazení programu. Kdykoli se na něco potřebujete kódu zeptat můžete tak učinit tímto prostým příkazem.

V konzoli se tisk projevuje bílou barvou. Další možný obsah konzole jsou chyby. Ty jsou označeny barvou červenou. Processing vám chybou naznačuje, že něco není v pořádku s vaším kódem. V případě takto jednoduchého programu se v drtivé většině případů bude jednat o překlep, či zapomenutý znak.

Programovací jazyk bohužel (naštestí ?) neodpouští žádné překlepy. Tedy bude stačit jeden chybný znak v programu aby celý program nebyl spustitelný.

Bohužel chybové hlášení se nedá označit za dokonalé a pro začátečníky bude velmi složité dobrat se pomocí chybových hlášení k původu chyby. Na zlepšení chybových hlášek se pracuje již dlouhou dobu a některé základní chyby processing v angličtině umí dobře popsat. Většinou ale Processing tiskne řetězec svých chyb, které byly nastartovány chybou vaší a chybová hlášení čítající několik desítek řádků vám toho nakonec o původní chybě moc nesdělí.

V případě chyby se vás Processing bude snažit přesměrovat na řádek, kde chyba pravděpodobně vznikla. V případě takto jednoduchých programů, jaké si zde ukazujeme, chybu identifikuje zcela bezchybně, bohužel tomu tak nebude ve všech případech.


\subsection{Základní operace s datatypy}

Náš program nyní nědělá nic světoborného. Definuje pár proměnných a následně některé z nich tiskne do konzole. Operujeme zde stále s abstraktními hodnotami které se zapisují do paměti programu a ty pak z paměti zpětně získáváme.

Nyní si ukážeme co s již definovanými proměnnými můžeme dělat. Jenotlivé datatypy mají různé přípustné operace. Zkusme si letmo projít možnosti našich proměnných.

\begin{itemize}


\item{Integer a Float, neboli čísla: {\em int a float}}

První proměnná {\em mojeCeleCislo}, které byla přiřazena hodnota {\em 1}, je takzvaný {\em Integer}, tedy celé číslo. Tento typ může mít hodnotu\footnote{na 32-bitových strojích} v rozsahu {\em -65575 až 65576} [upřesnit] . V rozmezí těchto hodnot můžeme provádět různé matematické operace mezi čísly:

\begin{lstlisting}
int prvniCislo = 1;
int druheCislo = 5;
int tretiCislo;

tretiCislo = prvniCislo + druheCislo;

println(tretiCislo);
\end{lstlisting}

Tímto jsme provedli základní matematickou operaci, sečetli jsme dvě čísla a následně jsme výsledek vytiskli do konzole.

Další možné operace jsou:

\begin{lstlisting}
// aritmetické operace
a + b;
a - b;
a * b;
a / b;

// přírůstky
a += b;
a -= b;

// přírůstek, úbytek o 1
a ++;
a --;

// modulo (přebytek po dělení)
a % b;

// a konečně logické porovnávání
// větší, menší
a < b;
a > b;

//větší nebo rovná se, menší nebo rovno
a <= b;
a >= b;

// shoda, neshoda
a == b;
a != b;

// pozor výsledkem porovnávání již není číslo!
// ale odpověd ANO nebo NE, TRUE nebo FALSE
// pravda nebo nepravda, datatyp boolean

\end{lstlisting}

Celá čísla neboli {\em int} nebudou mít problém se sčítáním, odčítáním a násobením celých čísel. V případě dělení může nastat logicky problém, výsledek nemůsí být celé číslo. Budeme-li chtít výsledek takové operace uchovat v paměti, tj. zapsat pod naši proměnnou, musíme použít datatyp operující s desetinou čárkou, nebo výsledek zaokrouhlit na celé číslo.

například:

\begin{lstlisting}
int a = 10;
int b = 3;
float c;

c = a / b;

println(c); 
\end{lstlisting}

Vytiskne {\em 3.0}. Pozor, to není správný výsledek! Kde se stala chyba? Processing operující s celými čisly, speciálně při dělení předpokldádá výsledek opět za celé číslo. Tedy ono zaokrouhlení provádí již sám. Zde si dovolím nesouhlasit s tvůrci, kteří se tímto snaží začátečníky vyvarovat chyb. Stačí si nyní pamatovat, že pro přesné dělení čísel bychom měli dělit vždy číslem s desetinnou čárkou, tj. s datatypem {\em float}.

Tedy správně by dělení mělo proběhnout takto:

\begin{lstlisting}
float a = 10;
float b = 3;
float c;

c = a / b;

println(c); 
\end{lstlisting}

Tisk do konzole již ukazuje správnou hodnotu {\em 3.333333} .

Pro další operace s čísly bych pro přesnost výsledků důrazně doporučil používat jen float. Další operace mohou vypadat například takto:

\begin{lstlisting}
float a = 3;
float b;

// sq je funkce pro "square", číslo na druhou
b = sq(a);
println(b);

// sqrt je funkce pro "square root", odmocninu
b = sqrt(a);
println(b);

// pow je funkce pro "power", číslo na N-tou
// mínusová čísla v N jsou odmocniny
b = pow(a,3);
println(b);

b = pow(a,-3);
println(b);

// atp.
\end{lstlisting}

\item{{\em char a String}, znak a řetězec znaků, text}

S textem nelze operovat stejně jako s čísly, je logické, že nemůžeme násobit texty mezi sebou. {\em String} je speciální datatyp pro uchovávání textů v paměti a nakládá se s ním speciálními funkcemi. Prozatím nám bude stačit, když si ukážeme velmi jednoduchou operaci s textem, spojení dvou řetězců dohromady.

Pro označení hodnoty řetězce se používají dvojité uvozovky. S textem se pracuje následovně:

\begin{lstlisting}
String prvniSlovo = "Ahoj";
String druheSlovo = "světe!";
String slovniSpojeni;

slovniSpojeni = prvniSlovo + " " + druheSlovo;

println("Dvě spojená slova: " + sovniSpojeni);
\end{lstlisting}

Všimněte si prosím vloženeé mezery mezi slova " ". Uvozená mezera je počítána také jako řetězec textu. Výsledným tiskem do konzole tedy dostaneme následující řetězec textu:

\begin{lstlisting}
Dvě spojená slova: Ahoj světe!
\end{lstlisting}

Zde jsme provedli jednu ze základních operací s textem, spojování řetězců. {\em String} lze také spojit s jednotlivými znaky, nebo také s čísly, výsledkem bude ovšem vždy další {\em String}.

\begin{lstlisting}
int a = 1;
int b = 2;
String slova = "test";

// "slova = slova + něco" lze také nahradit znaménkem "+="
// stejným znaménkem které u čísel znamená přírůstek
// tedy namísto:
// slova = slova + " " + a + " " + b;
// můžeme zkrátit na:

slova += " " + a + " " + b;

println(slova);
\end{lstlisting}

Výsledkem bude: {\em test 1 2}.

Řetězce se dají dále porovnávat seřazovat, dá se v nich vyhledávat znak či slovo a tak podobně. Pro náše účely zatím postačí si uvědomit rozdílné operace mezi textem a číslem.


\item{{\em boolean}, neboli pravda nebo nepravda}

Boolean je nejjednodušším datatypem v Processingu, může vyjadřovat pouze dva stavy. Pravdu {\em true}, nebo nepravdu {\em false}. Žádnou jinou hodnotu {\em boolean} nepřijímá a je proto  ,co se paměti týče, velmi úsporný datatyp. Operace s booleany se nazývají logické operace a vypadají následovně:


\begin{lstlisting}
boolean prvniTvrzeni = true;
boolean druheTvrzeni = false;
boolean tretiTvrzeni;

// "&&" značí logickou operaci AND
// výsledek bude TRUE, pravda, JEN pokud
// prvniTvrzeni a druheTvrzeni bude pravda
tretiTvrzeni = prvniTvrzeni && druheTvrzeni;

println(tretiTvrzeni);

// "||" dvě svislé čáry je OR, tj "nebo"
// výsledek bude TRUE, pravda pokud
// prvniTvrzeni nebo druheTvrzeni je pravda
tretiTvrzeni = prvniTvrzeni || druheTvrzeni;

println(tretiTvrzeni);

// posledni zakladni operaci je porovnani
// muzeme porovnat dva booleany pomoci "=="
// dvojiteho rovnitka
tretiTvrzeni = (prvniTvrzeni == druheTvrzeni);

println(tretiTvrzeni);

// pro negativni vysledek je zaporne porovnani "!="
// vykricnik pred booleanem vzdy znaci opak

tretiTvrzeni = (prvniTvrzeni != druheTvrzeni);

println(tretiTvrzeni);

\end{lstlisting}

Boolean je možné si představit jako vypínač světla. Zapnuto a vypnuto jsou jediné dva stavy podobného vypínače. Booleany často řídí tok programu, je možné si je představit jako přepínače mezi jednotlivými stavy programu.

\end{itemize}

\subsection{Podmínka}

Uvedením do stavů programu se můžeme dostat k první opravdové struktuře v programu. Zkusme nyní nastínit jak taková struktura vypadá. Patrně nejpřímější řízení dějů v programu je podmínka. Podmínka říká, jestliže je něco pravda, spust následové příkazy. Uvedu zde krátký příklad podobné struktury:

\begin{lstlisting}
boolean prepinac = false;
int hodnota;

if(prepinac == true){
    hodnota = 1;
}else{
    hodnota = 0;
}

println(hodnota);

\end{lstlisting}

Velmi prostá struktura je v tomto příkladě vytvořená jednou podmínkou. Podmínka je v Processingu (a řadě jiných jazyků) značena slovem {\em if}, "jestli". "Jestli" potřebuje dostat odpověd v kulatých závorkách, ta může být jen pravda nebo nepravda, k tomu se hodí nejlépe již zmíněný {\em boolean}, nebo například porovnávání dvou čísel, znaků nebo řetězců znaků, tj. operace které nám vrátí {\em TRUE} nebo {\em FALSE}.

Definujeme-li tedy náš {\em boolean} na začátku jako nepravdu, podmínka se v tomto případě nenaplní a program spustí kód uvozený složenými závorkami následujícími až slovo {\em else}.
V našem příkladě se jedná o druhou větev podmínky, ta sice není povinná ale pro ukázku jsem ji zde rovnou zmínil. Tedy za ukončením {\em if} a složených závorek můžeme dále slovem {\em else}, tzn. "jestliže ne", řící co se stane v případě nenaplnění naší podmínky.

V tomto konkrétním případě se k proměnné {\em hodnota} přiřadí číslo {\em 0}.

To program následně ověřuje tiskem do konzole, kde se objeví {\em 0}.

\section{Hodnota a její zobrazení}

K čemu jsou vlastně hodnoty dobré? Tisk do konzole je jen kontrolní mechanismus většinou se nejedná o výslednou podobu programu.

Po celou dobu sčítání a odčítání hodnot jsme nezavolali žádnou funkci, která by kreslila na plátno.

Nyní je na čase ukázat si jakým způsobem Processing rozumí kresbě. Ty samé hodnoty které máme nyní v paměti mohou být použity pro jakýkoli kreslící výstup. Řekněme že chceme z těchto hodnot zobrazit například elipsu. K tomu potřebujeme zavolat funkci pro tvorbu elipsy, pokud ji nechceme zrovna z nějakých důvodů popisovat matematicky (to je samozřejmně dokonale možné).

Processing nemá předdefinované žádné složité tvary. Pracuje sám o sobě pouze s tvary primitivními, jako je bod, linka, trojuhelník, obdélník a elipsa. Pomocí těchto tvarů lze zkonstruovat nepřeberné množství obrazů. Představíme-li si nyní digitálně zpracovanou fotografii, můžeme například říci že je zkonstruována z bodů. Jednotlivé body, tj. pixely mají jinou barevnou hodnotu a takto poskládaný obraz se ve výsledku jeví jako fotografie.

Problém v případě konstrukce syntetické fotagrafie nespočívá v geometrii; ta je známa, mřížka bodů s počtem šířky krát výšky obrazových bodů digitální fotografie. Problém je v barevných hodnotách, které neznáme a synteticky jakoukoli matematickou funkcí těžko obsáhneme.

Tento příklad je trochu extrémní, ale naprosto pravdivý. Věc kterou se snažím ilustrovat je ta, že i s minimálním počtem primitivních geometrických tvarů lze docílit tařka nekonečné (spočitatelně obrovské) množství obrazů, což by nám mělo nadlouho stačit.

Nyní zpět k hodnotám. Máme-li již jakékoli hodnoty v paměti programu, můžeme kterékoli z nich namapovat na jakoukoli kreslící funkci. Zde začíná naše experimentální část, často se totiž stává, že výsledek kreslení neumíme plně předpovědět a teprve zobrazením nám kresba vzhledem k hodnotě začne dávat smysl.

Processing je, co se týče grafického výstupu, navržen pro přímou experimentální interakci vznikajícího programu s uživatelem. To znaměná, že pokaždé kdy chete vidět výsledek kódu, stačí pouze stisknout tlačítko {\em RUN}. Zpětná vazba spolu s vaší interpretací a úmyslem má potom veliký vliv na následné úpravy v kódu. Tímto způsobem můžete doslova "vysochat" výslednou podobu programu.

\subsection{Zobrazení}

Jak zobrazit hodnoty, které uchovává program v paměti? Jakákoli hodnota lze použít například jako jeden z argumentů v kreslících funkcích.

Již zminovaný {\em rect}, tedy rectangle - obdélník, vyžaduje ke svému úspěšnému vykreslení čtyři parametry, čtyři číselné hodnoty \footnote{zde nezáleží jestli se jedná o celé číslo int nebo číslo s destinnou čárkou float}. Tyto parametry, hodnoty, můžeme budto zadat jako přímé číselné hodnoty, nebo do těchto parametrů můžeme zadat naši proměnnou která obsahuje číslo. Názorně:

\begin{lstlisting}

// první způsob vykreslení obdéníku
rect( 10, 8, 40, 20 );

// druhý způsob vykreslení obdélníku
int prvniCislo = 10;
int druheCislo = 8;

rect( prvniCislo, druheCislo, 40, 20 );

\end{lstlisting}


Již nyní je nám zřemé jak hodnoty mohou ovlivnit zobrazení. Namísto hodnot zadaných čísleně se zde v druhém způsobu vykreslení obdélníku objevují naše proměnné, které mají již zadaný obsah. Výsledek zobrazení bude v obou případech stejný, ovšem z hlediska struktury programu je druhý způsob daleko flexibilnější.

Představme si nyní, že hodnota zadaná prvním způsobem zápisu se již nikdy v průběhu programu nemůže proměnit. V případě druhého způsobu zápisu získáváme díky možnosti změny jednoho parametru kontrolu nad kreslícím výstupem.

Veškeré příkazy které jsme si zatím ukázali byli spušteny pouze jednou. To znamená, že obdélník byl vykreslen a program, který již neměl žádné další příkazy jednoduše skončil.

V další kapitole si předvedeme jak vdechnout procesům pohyb a stálost, další kapitola je věnovaná animaci.


\section{Animace a interakce}

Animace, neboli-li vdechuntí žívota, se v processingu provádí opakovaným kreslením na plátno.

\newpage
\section{Datatypy, paměť programu}
.
\newpage
\section{Proměnlivost}
.
\newpage
\section{Pokročilejší logické operace}
.
\newpage
\section{Parsing, získávání hodnot z externích dat}
.
\newpage
\section{Vizualizace hodnot}
.
\newpage
\section{Knihovny a rozšíření}
.
\newpage
\section{Komplexní program}
.
\newpage
\section{Experimenty}


\newpage
\section{Rejstřík pojmů}

\printindex



\end{document}
