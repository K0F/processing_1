\newglossaryentry{acos()}{
	name={acos()},
	description={Inverzní funkce \vyraz{cos()}, arkuskosinus. Funkce převádí hodnoty rozmezí od -$1{,}0$ do $1{,}0$ na hodnoty v rozmezí od $0$ do $\pi$}
}
\newglossaryentry{asin()}{
	name={asin()},
	description={Inverzní funkce \vyraz{sin()}, arkussinus. Funkce převádí hodnoty rozmezí od $1{,}0$ do -$1{,}0$ na hodnoty v rozmezí od $0$ do $\pi$ {\em (3,1415927)}}
}
\newglossaryentry{atan2()}{
	name={atan2()},
	description={Vypočítává úhel z jednoho bodu do bodu druhého. K výpočtu můžeme použít zápis {\em atan2(}$Y_2-Y_1$,$X_2-X_1${\em )}. Výsledný úhel je udáván v radiánech, tedy v rozmezí od -$\pi$ do $\pi$. {\em Atan2()} je funkce, která se hojně využívá při určení směru od objektu ke kurzoru nebo jinému bodu. Pozor při zadávání parametrů. Všimněte si obráceného pořadí parametrů. Oproti zvyklosti jsou zadávány v pořadí $Y$, $X$}
}
\newglossaryentry{background()}
{
  name={background()},
  description={Funkce nastavuje barvu pozadí na plátně. Standardní barva je světle šedá. Spuštěním této funkce s definicí barvy v kulatých závorkách bude vyplněna celá plocha jednolitou barvou}
}
\newglossaryentry{beginRecord()}{
	name={beginRecord()},
	description={Otevírá nový soubor pro zápis do souboru ve formátu {\em PDF} nebo {\em DXF}. Kresba tak probíhá souběžně na plátno i do souboru. Pro ukončení kresby musíme spustit funkci \vyraz{endRecord()}}
}
\newglossaryentry{beginShape()}{
	name={beginShape()},
	description={Pomocí funkcí {\em beginShape()} a \vyraz{endShape()} můžeme definovat složitější tvary. Funkce {\em beginShape()} otevírá definici tvaru. Následuje funkce \vyraz{vertex()}, která určuje jednotlivé body tvaru. Tvar je nakonec uzavřen příkazem \vyraz{endShape()}. Kresba tvarů může probíhat ve dvou i třech rozměrech. V závislosti na renderu můžeme definovat \vyraz{vertex()} dvěma nebo třemi hodnotami. Doplňujícím parametrem v kulatých závorkách funkce {\em beginShape()} dále může být uveden způsob, jakým budou body spojovány. Přípustné varianty jsou: {POINTS, LINES, TRIANGLES, TRIANGLE\/FAN, TRIANGLE\/STRIP, QUADS a QUAD\/STRIP}}
}
\newglossaryentry{boolean}
{
  name={boolean},
  description={Datatyp, který může mít jen dva stavy: \vyraz{true} a \vyraz{false}}
}
\newglossaryentry{box()}{
	name={box()},
	description={Krychle nebo kvádr. Krychli můžeme definovat jedním parametrem, který určuje délku jedné hrany. Kvádr je možno definovat stejným příkazem zadáním délky tří hran: $X$, $Y$, $Z$}
}
\newglossaryentry{Built with Processing}
{
	name={Built with Processing},
	description={Doslova znamená: \uv{Postaveno s {\em Processingem}}. \linebreak Jedná se o~zvláštní komunitní frázi, která se objevuje u projektů vy\-užívajících {\em Processing}. Fráze vyjadřuje určitý vděk všem participantům a~tvůrcům {\em Processingu} za tvorbu tohoto nástroje}
}
\newglossaryentry{camera()}{
	name={camera()},
	description={Nastavuje pozici kamery skrze parametry: pozice oka, střed zobrazení a osa označující směr nahoru. Funkce zavolaná bez jakýchkoli parametrů nastaví kameru na výchozí pozici. Výchozí pozice lze vyjádřit následovně: {\em camera(width/2.0, height/2.0, (height/2.0) / tan(PI*60.0 / 360.0), width/2.0, height/2.0, 0, 0, 1, 0)}}
}
\newglossaryentry{class}{
	name={class},
	description={Takzvaná třída. Slovo označující deklaraci nové třídy, definici \linebreak objektu. Třída je souhrn funkcí a proměnných a představuje šablonu pro objekt. Jména tříd jsou většinou označována počátečním velkým písmenem, instance objektů (produkt ze šablony) pak písmenem malým. Třída je základní stavební kámen objektově orientovaného programování}
}
\newglossaryentry{color()}{
	name={color()},
	description={Vytváří barvu ve speciálním datatypu {\em color}. Vstupní hodnoty jsou interpretovány buď jako hodnoty RGB (červená, zelená, modrá),\linebreak nebo HSB (odstín, sytost, jas). Jednotlivé mody vytváření barev lze přepínat funkcí \vyraz{colorMode()}. Ve výchozím nastavení mají barvy rozmezí 256 odstínů šedi, tedy konkrétně od $0$ do $255$}
}
\newglossaryentry{colorMode()}{
	name={colorMode()},
	description={Funkce měnící způsob, jakým {\em Processing} interpretuje barvu. Možné argumenty jsou RGB a HSB. Nepovinný je druhý argument, určující rozsah hodnot. Výchozí nastavení této hodnoty je $255$. Příkaz ovlivňuje nastavení barev pro kresbu, tj. například pro funkce \vyraz{fill()} a \vyraz{stroke()}}
}
\newglossaryentry{cos()}{
	name={cos()},
	description={Počítá hodnotu kosinu z úhlu. Tato funkce očekává parametr vyjadřující úhel, parametr může být v rozsahu od $0$ do $\pi*2$. Hodnoty se pak pohybují mezi -$1$ do $1$}
}
\newglossaryentry{CSV}{
	name={CSV},
	description={\hspace{0.2mm} Neboli \uv{Comma Separated Values}. Jedná se o uznávaný standard uložených dat.  S tímto standardem  zachází mnoho programů. Jde v~podstatě o textový soubor s informacemi oddělenými specifickým znakem. Nejčastěji je tímto specifickým znakem čárka. Rozlišovacím znakem může ovšem být například středník nebo teoreticky jakýkoli jiný znak}
}
\newglossaryentry{draw()}{
	name={draw()},
	description={Základní kreslicí smyčka programu. Kód uvozený ve složených závorkách za touto funkcí bude spouštěn několikrát ve vteřině. Využívá se k animaci a interakci uživatele s programem}
}
\newglossaryentry{ellipse()}{
	name={ellipse()},
	description={Kreslí elipsu nebo kruh. Elipsa se stejnou šířkou i výškou představuje kruh. První dva parametry udávají pozici, třetí uvádí šířku a~čtvrtý výšku elipsy. Nastavení způsobu vytváření elips lze provést pomocí funkce \vyraz{ellipseMode()}}
}
\newglossaryentry{ellipseMode()}{
	name={ellipseMode()},
	description={Ovlivňuje způsob, jakým je vykreslována elipsa. Možné \linebreak mody jsou {\em RADIUS, CENTER, CORNER, CORNERS}. Výchozí modem je {\em CENTER}. První dva parametry nastavují pozici elipsy, druhé dva pak šířku a~výšku elipsy. V případě modu {\em RADIUS} druhé dva parametry určují poloměry elipsy. Modus {\em CORNER} definuje pozici levého horního bodu, následně šířku a~výšku. Poslední modus {\em CORNERS} definuje dva body, mezi které bude elipsa vepsána}
}
\newglossaryentry{else}{
	name={else},
	description={\hspace{0.5mm} Doplňuje podmínku \vyraz{if} -- \uv{jestliže} o \uv{jestliže ne}. Blok kódu za tímto příkazem je spuštěn, jestliže není předchozí \vyraz{if()} podmínka splněna. Struktura kódu tím dovoluje spustit dva různé bloky kódu v případě \\ naplnění podmínky anebo jejího nenaplnění}
}
\newglossaryentry{endRecord()}{
	name={endRecord()},
	description={Zastavuje nahrávání do souboru spuštěné pomocí funkce \vyraz{beginRecord()}, dále uzavírá zápis do souboru {\em (EOF)}}
}
\newglossaryentry{endShape()}{
	name={endShape()},
	description={Párová funkce s \vyraz{beginShape()}, uzavírá kreslení tvaru. Funkce může být doplněna o výraz {\em CLOSE} (psáno dovnitř kulatých závorek), který uzavře tvar a pokusí se vytvořit jeho výplň}
}
\newglossaryentry{exit()}{
	name={exit()},
	description={Ukončí program. Program bez funkce \vyraz{draw()} bude ukončen automaticky poslední řádkou kódu. Programy obsahující funkci \vyraz{draw()} běží, dokud není program zvenčí ukončen nebo není zavolán příkaz {\em exit()}}
}
\newglossaryentry{false}
{
  name={false},
  description={Nepravda neboli 0}
}
\newglossaryentry{fill()}{
	name={fill()},
	description={\hspace{0.45mm} Funkce nastavující barvu výplně tvarů. Například spustíme-li {\em fill(204, 102, 0)}, veškeré objekty, kreslené níže v kódu, budou mít oranžovou barvu výplně}
}
\newglossaryentry{flow}
{
  name={flow},
  description={Neboli plynutí, tok, je zvláštním stavem mysli popisovaným programátory. Jedná se o stav, kdy je člověk plně zanořen do práce. Jakékoli vyrušení z tohoto stavu si vyžaduje dlouhou koncentraci k opětovnému nastolení bdělosti. Podle zkušených programátorů vyvolání takového stavu trvá několik desítek minut práce s kódem}
}
\newglossaryentry{for}{
	name={for},
	description={\hspace{2mm} Definuje smyčku, tedy opakující se sekvenci příkazů. Struktura \linebreak {\em for} se zadává pomocí tří parametrů: inicializace pozice (kde smyčka začíná), ověření (konečná podmínka, která smyčku ukončuje) a přírůstku (jeden krok rozdílu při opakování smyčky). Tyto části jsou vždy odděleny středníkem. Smyčka pokračuje, dokud je naplněn výsledek prostřední podmínky, tedy \vyraz{true}}
}
\newglossaryentry{frameCount}
{ 
  name={frameCount},
  description={Proměnná uchovávající údaj o počtu vykreslených okének od startu programu}
}
\newglossaryentry{frameRate()}{
	name={frameRate()},
	description={Nastavuje počet okének, které mají být vykresleny za jednu vteřinu. Výchozí nastavení je šedesát okének za vteřinu. Pokud je údaj vyšší, než {\em Processing} může v danou chvíli vykreslovat, počet okének za vteřinu se sníží na maximální možnou hodnotu}
}
\newglossaryentry{GNU / GPL}
{
  name={GNU / GPL},
  description={Jedna z licencí otevřeného softwaru zaručující volně dostupný zdrojový kód. Základním konceptem licence je podmínečná volná dostupnost zdrojového kódu takto licencovaného programu. A dále nutnost použití stejné licence pro veškeré projekty takový kód využívající. Licencí zajišťujících otevřenost softwaru je nepočitatelně, mezi oblíbené například patří i méně restriktivní {\em BSD / MIT} licence, pod kterou je vyvíjen například i programovací jazyk -- prostředí {\em Processing}}
}
\newglossaryentry{HALFPI}{
	name={HALFPI},
	description={Konstanta určující polovinu čísla $\pi$, 3,14159265 }
}
\newglossaryentry{image()}{
	name={image()},
	description={Zobrazuje obrázek v datatypu \vyraz{PImage}. K jejich načtení používáme funkci \vyraz{loadImage()}. Barvu a průhlednost obrázku lze dále definovat pomocí funkce \vyraz{tint()}. Funkce {\em image()} jako první přijímá parametr s názvem proměnné obrázku, který bude zobrazovat. Zbylé dva parametry jsou pozice vykreslení obrázku v ose $X$ a $Y$. Ve výchozím modu souřadnice určují levý horní roh obrázku. Třetí a čtvrtý parametr je nepovinný, může dále nastavit šířku a výšku zobrazovaného obrázku. Mody pro alternativní zobrazování lze přepínat pomocí funkce \vyraz{imageMode()}}
}
\newglossaryentry{imageMode()}{
	name={imageMode()},
	description={Ovlivňuje způsob zobrazování obrázků. Dovoluje použít \linebreak konstanty {\em CORNER}, {\em CORNERS} a {\em CENTER}. Ve výchozím nastavení {\em CORNER} bude obrázek umístěn levým horním rohem na definované souřadnice $X$ a $Y$. Modus {\em CORNERS} mění chování druhých dvou parametrů, které místo relativní šířky a výšky nastavují absolutní hodnoty v~osách $X$ a $Y$. Poslední modus {\em CENTER} umístí obrázek na střed definovaných souřadnic, zbylé dva parametry opět udávají relativní šířku a výšku obrázku}
}
\newglossaryentry{indenting}
{
  name={indenting},
  description={Neboli zarovnávání kódu do úhledných paragrafů. Zarovnávání slouží k rychlejší orientaci programátora ve struktuře kódu}
}
\newglossaryentry{interakce}
{
  name={interakce},
  description={{\em (lat. interactio od inter-agere, jednat mezi sebou)} Znamená vzájemné působení, jednání, ovlivňování všude tam, kde se klade důraz na vzájemnost a oboustrannou aktivitu na rozdíl od jednostranného, například kauzálního působení}
}
\newglossaryentry{keyPressed()}{
	name={keyPressed()},
	description={Funkce je spuštěna pokaždé, je-li stisknuta libovolná klávesa na klávesnici}
}
\newglossaryentry{keyReleased()}{
	name={keyReleased()},
	description={Funkce spuštěná tehdy, uvolníme-li libovolnou stisknutou klávesu}
}
\newglossaryentry{loadImage()}{
	name={loadImage()},
	description={Načítá obrázek do proměnné ve formátu \vyraz{PImage}. V kulatých závorkách se definuje cesta a název obrázku (včetně přípony) ve formátu \vyraz{String}. Obrázky by měly být umístěny v adresáři {\em DATA}.  {\em Processing} podporuje grafické formáty {\em PNG, GIF, JPEG a TGA}. Průhlednost obrázků je podporována pouze ve formátech {\em GIF}, {\em TGA} nebo {\em PNG}}
}
\newglossaryentry{millis()}{
  name={millis()},
  description={Funkce udávající počet uplynulých milisekund od startu programu. Na rozdíl od proměnné \vyraz{frameCount} tato funkce udává precizní \linebreak časový údaj od startu programu. Výsledná hodnota nezávisí na počtu vykreslených okének za vteřinu. Výsledný údaj tak lze využít například pro přesné časování animace, a~to bez ohledu na počet okének za vteřinu. V případě ukládání animací do videa je pro časování animace naopak téměř nepoužitelný, především kvůli zpomalení animace při ukládání okének}
}
\newglossaryentry{mouseButton}{
	name={mouseButton},
	description={Pomocí této funkce {\em Processing} automaticky registruje, které tlačítko myši bylo stisknuto. Jestliže bylo stisknuto levé tlačítko, tato systémová proměnná bude mít hodnotu {\em LEFT}. V případě pravého tlačítka bude mít proměnná hodnotu {\em RIGHT}. U prostředního tlačítka pak hodnotu {\em CENTER}}
}
\newglossaryentry{mousePressed()}{
	name={mousePressed()},
	description={Funkce je spuštěna pokaždé je-li stisknuto tlačítko myši. K následné detekci tlačítka můžeme použít proměnnou \vyraz{mouseButton}}
}
\newglossaryentry{mouseX}
{
  name={mouseX},
  description={Proměnná, která udává pozici kurzoru na plátně processingového okna v ose $X$}
}
\newglossaryentry{mouseY}
{
  name={mouseY},
  description={Proměnná, jež udává pozici kurzoru na plátně processingového okna v ose $Y$}
}
\newglossaryentry{nf()}{
	name={nf()},
	description={\hspace{1.0mm} Nástroj na formátování čísel do patřičného tvaru v textu. Přijímá parametr celého čísla a předřazuje před něj počet nul z druhého parametru. Tato funkce se velmi často používá k zarovnávání řad čísel}
}
\newglossaryentry{noFill()}{
	name={noFill()},
	description={Zakazuje výplň všech následujících objektů. Jestliže je příkaz použit dohromady s příkazem  \vyraz{noStroke()}, {\em Processing} teoreticky nebude nic vykreslovat na plátno}
}
\newglossaryentry{noise()}{
	name={noise()},
	description={Udává hodnotu {\em Perlinova šumu} na specifické pozici. {\em Perlinův šum} je generátor pseudonáhodných čísel vytvářející plynule přecházející pseudonáhodnou hodnotu. Algoritmus byl vyvinut Kenem Perlinem v osmdesátých letech 20. století a je velmi často používán právě v~grafických aplikacích. Používá se například při tvorbě procedurálních textur, pohybu objektů nebo proměně geometrie v čase. Rozsah {\em Perlinova šumu} se pohybuje vždy v rozmezí hodnot $0{,}0$ \linebreak a~$1{,}0$. Funkce {\em noise()} přijímá jednu, dvě nebo tři hodnoty. Výsledná hodnota se pohybuje v jedno- až trojrozměrném poli. Šum může být například snadno animován přírůstkem v parametru. Hodnoty mají tu vlastnost, že plynule přecházejí jedna v druhou}
}
\newglossaryentry{noSmooth()}{
	name={noSmooth()},
	description={Opak funkce \vyraz{smooth()}. Vypíná vyhlazování hran}
}
\newglossaryentry{otevřený software}
{
  name={otevřený software},
  description={Také nazývaný svobodný. V zásadě software s veřejně dostupným zdrojovým kódem (viz \vyraz{GNU / Linux})},
  plural={otevřeného softwaru}
}
\newglossaryentry{parseFloat()}{
	name={parseFloat()},
	description={Funkce pro čtení čísel ve formátu (\vyraz{String}) a získávání číselné hodnoty s desetinnou čárkou ve tvaru {\em float}}
}
\newglossaryentry{parseInt()}{
	name={parseInt()},
	description={Funkce pro čtení čísel ve formátu (\vyraz{String}) a získávání celých čísel ve tvaru {\em int}}
}
\newglossaryentry{PI}{
	name={PI},
	description={\hspace{4mm} Konstanta určující číslo $\pi$, 3,14159265}
}
\newglossaryentry{point()}{
	name={point()},
	description={Kreslí bod v dvourozměrném zobrazení. Funkce vyžaduje parametry $X$, $Y$. Při trojrozměrném zobrazení bodu funkce vyžaduje tři parametry pro $X$, $Y$ a $Z$. Barvu bodu můžeme kontrolovat pomocí příkazu \vyraz{stroke()}}
}
\newglossaryentry{print()}{
	name={print()},
	description={Funkce pro tisk do konzole. Vstupní hodnotou může být jakákoli proměnná, holý text, nebo kombinace obojího. Slouží jako základní nástroj k sledování proměn v programu. Využívá se zejména při vylaďování programu a při kontrole výsledků}
}
\newglossaryentry{println()}{
	name={println()},
	description={Funkce pro tisk do konzole. Funguje stejně jako příkaz \vyraz{print()}, \linebreak s~tím rozdílem, že každou tisknutou hodnotu zakončuje novým \linebreak řádkem}
}
\newglossaryentry{pushMatrix()}{
	name={pushMatrix()},
	description={Ukládá nastavení koordinačního systému (neboli matrix). \linebreak Funkce dovoluje vrstvení jednotlivých transformačních operací, jako je \vyraz{rotate()}, \vyraz{translate()} nebo \vyraz{scale()}. Funkce vyžaduje následné ukončení transformace příkazem \vyraz{popMatrix()}}
}
\newglossaryentry{popMatrix()}{
	name={popMatrix()},
	description={Funkce načítá z uloženého koordinačního systému a uzavírá blok transformací. Vyžaduje uložení koordinátů předchozím vyvoláním příkazu \vyraz{pushMatrix()}}
}
\newglossaryentry{quad()}{
	name={quad()},
	description={Funkce kreslící čtyřúhelník. Nejvíce je použitelná pro lichoběžníky. Funkce vyžaduje parametry $X_1$, $Y_1$, $X_2$, $Y_2$, $X_3$, $Y_3$, $X_4$ a $Y_4$. Barvu čtyřúhelníku můžeme kontrolovat pomocí příkazu \vyraz{stroke()}. Barvu výplně definujeme funkcí \vyraz{fill()}}
}
\newglossaryentry{QUARTERPI}{
	name={QUARTERPI},
	description={Konstanta určující čtvrtinu čísla $\pi$, 3,14159265 }
}
\newglossaryentry{random()}{
	name={random()},
	description={Funkce, která vrací pseudonáhodné číslo. Není-li zadána žádná hodnota do kulatých závorek, výsledek funkce se pohybuje v rozmezí mezi $0 - 1$. Je-li zadána jedna vstupní hodnota, číslo se pohybuje mezi nulou a zadanou hodnotou. Jsou-li zadány dvě hodnoty oddělené čárkou, výsledek se bude pohybovat v rozmezí těchto dvou hodnot}
}
\newglossaryentry{randomSeed()}{
	name={randomSeed()},
	description={Funkce udává takzvanou hodnotu {\em seed} pro všechny následující funkce \vyraz{random()}. Nezadáme-li hodnotu {\em seed}, funkce \vyraz{random()} bude pokaždé generovat jedinečný výsledek}
}
\newglossaryentry{rect()}
{
  name={rect()},
  description={Funkce vykresluje obdélník, přičemž přijímá čtyři parametry: počátek $X$, počátek $Y$, šířku a výšku. Možné způsoby vykreslování \linebreak obdélníka se dají nastavit funkcí \vyraz{rectMode()}}
}
\newglossaryentry{rectMode()}
{
  name={rectMode()},
  description={Funkce ovládá modus vykreslování obdélníka. Možné vstupní parametry jsou pouze čtyři:  {\em CORNER} je základní chování funkce \vyraz{rect()}. Tedy prvními dvěma parametry při kresbě umístíme nejdříve první roh v ose $X$ a $Y$. Poté definujeme relativní šířku a výšku obdélníka. Druhou možností je {\em CORNERS}. Tento modus definuje dva rohy zvlášť, tedy v absolutních číslech na ploše. Dalším modem je {\em CENTER}. V tomto modu jsou obdélníky vystředěny kolem bodů $X$ a $Y$. Pomocí dvou dalších parametrů opět funkce \vyraz{rect()} definuje šířku a~výšku tohoto obdélníka. Posledním modem je {\em RADIUS}. Ten nastavuje opět středové body $X$ a $Y$ v prvních dvou parametrech. Druhé dva parametry udávají polovinu šířky a polovinu výšky obdélníka}
}
\newglossaryentry{rotate()}{
	name={rotate()},
	description={Provádí rotaci plátna v dvourozměrném zobrazení. Střed rotace je vždy aktuální souřadnice $X = 0$ a $Y = 0$}
}
\newglossaryentry{rotateX()}{
	name={rotateX()},
	description={Funkce provádí rotaci celým prostorem kolem osy $X$. Vyžaduje hodnotu udanou ve stupních -- radiánech. Hodnota se pohybuje v rozmezí $0$ až $2 * \pi$}
}
\newglossaryentry{rotateY()}{
	name={rotateY()},
	description={Funkce provádí rotaci celým prostorem kolem osy $Y$. Vyžaduje hodnotu udanou ve stupních -- radiánech. Hodnota se pohybuje v rozmezí $0$ až $2 * \pi$}
}
\newglossaryentry{rotateZ()}{
	name={rotateZ()},
	description={Funkce provádí rotaci celým prostorem kolem osy $Z$. Vyžaduje hodnotu udanou ve stupních -- radiánech. Hodnota se pohybuje v rozmezí $0$ až $2 * \pi$}
}
\newglossaryentry{save()}{
	name={save()},
	description={Pomocí příkazu save() ukládáme plátno programu do obrázku. Příkaz vyžaduje cestu k souboru. Cestu lze zadat buď úplnou (absolutní), nebo jen pouhý název souboru. Zadáním pouhého názvu se soubor uloží přímo do adresáře {\em sketche}. Příkaz dále rozlišuje mezi příponami automaticky. Obrázek tak uloží v patřičných formátech: TIF, JPG, PNG nebo TGA}
}
\newglossaryentry{saveStrings()}{
	name={saveStrings()},
	description={Pomocí příkazu {\em saveStrings()} ukládáme pole ve formátu \linebreak \vyraz{String}. Tento příkaz přijme dva parametry: název souboru a název našeho pole s textovou informací. Příkaz ukládá textový soubor na\-plněný hodnotami z pole. Jednu hodnotu zapisuje vždy na nový řádek}
}

\newglossaryentry{saveFrame()}{
	name={saveFrame()},
	description={Příkaz ukládá sérii obrázků. Stejně jako příkaz \vyraz{save()}. Funkce vyžaduje cestu k souboru. Speciální vlastností příkazu je možnost vložit za vstupní hodnotu \vyraz{String} symbol křížku, budou posléze detekovány a~nahrazeny číslováním sekvence okének. Každé číslo bude označovat číslo okénka podle proměnné}
}
\newglossaryentry{scale()}{
	name={scale()},
	description={Mění velikost geometrie vůči vykreslovacímu plátnu. Výchozí hodnota je $1.0$}
}
\newglossaryentry{setup()}
{
  name={setup()},
  description={Základní funkce pro nastavení výchozích parametrů programu. \linebreak Tato funkce je spuštěna pouze jednou, vždy na začátku běhu programu}
}
\newglossaryentry{sin()}{
	name={sin()},
	description={\hspace{0.2mm} Vypočítává sinus z daného úhlu. Tato funkce očekává údaj úhlu zadaný ve stupních radiánu, tedy hodnoty od $0$ do $6.28$. Zpětně funkce vrací hodnotu v rozmezí od -$1$ do $1$}
}
\newglossaryentry{size()}{
	name={size()},
	description={Nastavuje rozměry okna programu. Hodnoty jsou udávány v pixelech pro osu $X$ a $Y$. Tento příkaz by měl být spuštěn jako první ve funkci {\em setup()}. Jestliže není příkaz spuštěn, okno programu má výchozí velikost $100 * 100$ pixelů. Ve funkci {\em size()} je dále přípustný třetí parametr, ten je vyjádřen slovy: JAVA2D (výchozí render), \linebreak {\em P2D} (akcelerovaný 2D render), {\em P3D} (akcelerovaný 3D render), \linebreak {\em OPENGL} (3D akcelerovaný render)}
}
\newglossaryentry{sketch}
{
	name={sketch},
	description={Neboli {\em náčrt}, je koncept v prostředí {\em Processing} pro uchovávání jednotlivých projektů v adresářích. Jedna {\em sketch} je adresář, kam {\em Processing} ukládá data uživatele (tj. především kód a externí soubory)},
	plural={sketche}
}
\newglossaryentry{smooth()}{
	name={smooth()},
	description={Zapíná takzvaný anti-aliasing, tedy vyhlazování hran kresby. \linebreak Funkce nastavuje vyhlazování pro celý program. Zapnutí vyhlazování má za následek poměrně výrazné zpomalení kresby. Výstup kresby je ovšem o poznání jemnější}
}
\newglossaryentry{sphereDetail()}{
	name={sphereDetail()},
	description={Funkce nastavuje úroveň detailu při vykreslování objektu koule. Parametr zadaný ve formátu čísla \vyraz{float} ovlivňuje vykreslovaný počet stran koule}
}
\newglossaryentry{stroj}
{
  name={stroj},
  description={\hspace{0.1mm} Stroj je technické zařízení, které přeměňuje jeden druh energie nebo síly v jiný -- ať už kvalitativně, nebo kvantitativně. Původně byly stroje jen mechanické, ale dnes se tak označují i zařízení pracující na jiných fyzikálních či technických principech -- například elektrický transformátor. Strojem je v této knize téměř výhradně myšlen počítač. Počítač je programovatelný typ stroje, který přijímá vstup, ukládá a zpracovává data a umožňuje výstup v požadovaném formátu}
}
\newglossaryentry{stroke()}{
	name={stroke()},
	description={Funkce nastavující barvu obrysů tvarů. Například -- spustíme-li \linebreak {\em stroke(255, 0, 0)}, veškeré další kontury objektů budou mít červenou barvu}
}
\newglossaryentry{strokeWeight()}{
	name={strokeWeight()},
	description={Funkce nastavuje šířku vykreslovaných obrysů v~pixelech}
}
\newglossaryentry{syntax highlighting}
{
  name={syntax highlighting},
  description={Barevné odlišování slov podle jejich významů. \linebreak Odlišování slouží k lepší orientaci programátora v kódu}
}
\newglossaryentry{this}{
	name={this},
	description={\hspace{0.1mm} Speciální výraz vztahující se k objektu, ve kterém se zrovna pohybujeme. Slouží k identifikaci zejména pro práci s knihovnami nebo námi definovanými třídami}
}
\newglossaryentry{translate()}{
	name={translate()},
	description={Provádí pohyb celého plátna na zadané souřadnice. V trojrozměrném modu vykreslování požaduje třetí rozměr pro osu $Z$}
}
\newglossaryentry{vertex()}{
	name={vertex()},
	description={Funkcí {\em vertex()} definujeme body pro tvorbu složitějších tvarů. Funkci musí předcházet příkaz \vyraz{beginShape()} a musí ji následovat příkaz \vyraz{endShape()}, přičemž záleží na pořadí, ve kterém jsou jednotlivé body definovány. Body mohou být dvou- i trojrozměrné. {\em Vertex} tedy můžeme definovat pomocí souřadnic $X$, $Y$ nebo $X$, $Y$, $Z$}
}
\newglossaryentry{void}
{
  name={void},
  description={\hspace{0.1mm} Neboli prázdná funkce. Z anglického \uv{prázdno}. Funkce, která zpět nevrací žádný výsledek, pouze spouští sérii zadaných příkazů. Ty jsou uvozeny složenými závorkami}
}
\newglossaryentry{sphere()}{
	name={sphere()},
	description={Funkce kreslící kouli. Požaduje jediný argument: poloměr koule. Rozlišení koule je možné kontrolovat funkcí \vyraz{sphereDetail()}}
}
\newglossaryentry{float}{
	name={float},
	description={Datatyp schopný udržet proměnnou ve tvaru čísla s desetinnou čárkou}
}
\newglossaryentry{int}{
	name={int},
	description={Datatyp schopný udržet proměnnou ve tvaru celého čísla}
}
\newglossaryentry{CODED}{
	name={CODED},
	description={Používá se v porovnání se systémovou proměnnou \vyraz{key} k zjištění, zdali poslední stisknutá klávesa byla speciálním znakem. Klávesa se posléze detekuje proměnnou \vyraz{keyCode}}
}
\newglossaryentry{return}{
	name={return},
	description={Speciální příkaz značící návrat hodnoty z funkce. Příkaz zároveň ukončuje spuštěnou funkci}
}
\newglossaryentry{keyCode}{
	name={keyCode},
	description={Systémová proměnná {\em Processingu}, podobně jako \vyraz{key} zančí, která z netextových kláves byla stisknuta jako poslední ({\em ENTER, \linebreak BACKSPACE, LEFT atp.}). Velmi často se používá v~kombinaci \linebreak s~další systémovou funkcí \vyraz{keyPressed()} k detekci naposledy stisknuté klávesy}
}
\newglossaryentry{key}{
	name={key},
	description={\hspace{0.8mm} Systémová proměnná {\em Processingu} obsahující hodnotu naposledy \linebreak stisknuté klávesy, velmi často se používá v kombinaci s další systémovou funkcí \vyraz{keyPressed()} k detekci stisknuté klávesy}
}
\newglossaryentry{if}{
	name={if},
	description={\hspace{4mm} Příkaz značící podmínku. Vyžaduje následné uvedení dotazu v kulatých závorkách, pokud je podmínka naplněna, spouští následující kód uzavřený ve~složených závorkách. Komplementárním příkazem je \vyraz{else}, který může spustit další blok kódu, není-li naplněna předchozí podmínka.}
}
\newglossaryentry{if()}{
	name={if()},
	description={\hspace{2mm} Viz \vyraz{if}}
}
\newglossaryentry{PImage}{
	name={PImage},
	description={Proměnná schopná v sobě udržet obrazová data a obrázky obecně. Pro načtení dat do {\em Processingu} použijte \vyraz{loadImage()}}
}
\newglossaryentry{noStroke()}{
	name={noStroke()},
	description={Funkce zabraňující vykreslování kontur, jedná se o~protějšek funkce \vyraz{stroke()}}
}
\newglossaryentry{line()}{
	name={line()},
	description={Funkce pro kresbu úsečky. Ve 2D funkce přijímá čtyři parametry: $X_1$, $Y_1$,$X_2$, $Y_2$. Ve 3D zobrazení funkce přijímá šest parametrů pro určení koordinátů každého bodu ve třech dimenzích, tedy: $X_1$, $Y_1$,$X_2$, $Y_2$, $X_3$, $Y_3$. Barvu čáry je možné změnit předchozím nastavením barvy obrysů funkcí \vyraz{stroke()}}
}
\newglossaryentry{mouseClicked()}{
	name={mouseClicked()},
	description={Funkce spuštěná klikem myši. Celý jeden klik myši se skládá ze stisku tlačítka a jeho následného uvolnění}
}
\newglossaryentry{mouseDragged()}{
	name={mouseDragged()},
	description={Funkce detekující potáhnutí myši po plátně. Táhnutím myší se rozumí, podržení tlačítka myši a pohyb kurzoru}
}
\newglossaryentry{mouseMoved()}{
	name={mouseMoved()},
	description={Funkce detekující jakýkoli pohyb myši v rámci okna programu}
}
\newglossaryentry{noTint()}{
	name={noTint()},
	description={Resetuje barvu a průhlednost vykreslených obrázků}
}
\newglossaryentry{tint()}{
	name={tint()},
	description={Ovlivňuje barvu a průhlednost vykreslených obrázků. Protějškem je vyraz \vyraz{noTint()}}
}
\newglossaryentry{String}{
	name={String},
	description={Datatyp pro řetězec textu. Jedná se o speciální datatyp schopný uchovat více znaků pod jednou proměnnou. S tímto datatypem se pojí řada speciálních funkcí pro nakládání s textem. Představit si jej můžeme jako jednorozměrné pole znaků}
}
\newglossaryentry{triangle()}{
	name={point()},
	description={Kreslí trojúhelník. Vyžaduje parametry $X_1$, $Y_1$, $X_2$, $Y_2$, $X_3$ a $Y_3$. Barvu trojúhelníku můžeme kontrolovat pomocí příkazu \vyraz{stroke()}. Barvu výplně zadáváme pomocí funkce \vyraz{fill()}}
}
\newglossaryentry{radians()}{
	name={radians()},
	description={Převádí úhlové stupně do stupňů v radiánech. Tedy například hodnotu $0 - 360°$ převede na $0$ až $2 * \pi$}
}
\newglossaryentry{degrees()}{
	name={degrees()},
	description={Obrácená funkce \vyraz{radians()}. Převádí stupně v radiánech do úhlových stupňů}
}
\newglossaryentry{slovník}
{
  name={slovník},
  description={právě se zde nacházíte},
  plural={slovníky}
}
\newglossaryentry{true}
{
  name={true},
  description={\hspace{1mm} Pravda neboli 1}
}
\newglossaryentry{GNU / Linux}
{
  name={GNU / Linux},
  description={Zkratka \uv{Gnu Is not Unix}. {\em GNU} je projekt původně založený Richardem Stallmanem, jedná se o operační systém a rodinu programů s otevřeným zdrojovým kódem}
}
\newglossaryentry{MIT licence}
{
  name={MIT licence},
  description={Je licencí kompatibilní s licencí \vyraz{GNU / GPL}, jedná se o zvlášt\-ní licenci modifikovanou na půdě Univerzity MIT {\em -- Massachusetts \linebreak Institute of Technology} k možnému využití svobodného softwaru \linebreak i~v~komerčním prostředí},
  plural={MIT licencí}
}
\newglossaryentry{kompilace}
{
	name={kompilace},
	description={Proces překladu z čitelného textu do strojového kódu}
}
\newglossaryentry{zdrojový kód}
{
	name={zdrojový kód},
	description={Kód který je čitelný pro člověka},
	plural={zdrojové kódy}
}
\newglossaryentry{binarní kód}
{
	name={binární kód},
	description={Též strojový kód. Slovo vycházející z anglického slova {\em binary} -- dvojkové soustavy. V počítačové terminologii označuje kód srozumitelný pro procesor. Nejčastěji se s ním setkáme například v případě spustitelných aplikací},
	plural={binární kódy}
}
\newglossaryentry{Java}
{
	name={Java},
	description={Java je objektově orientovaný programovací jazyk, který vyvinula firma Sun Microsystems a představila jej 23. května 1995}
}
\newglossaryentry{CODE}
{
	name={CODE},
	description={Adresář uvnitř \slovnikpl{sketch}, který může obsahovat externí \slovnik{Java} a \vyraz{binární kód} programy nebo knihovny třetí strany}
}
\newglossaryentry{DATA}
{
	name={DATA},
	description={Adresář uvnitř \slovnikpl{sketch}, obsahuje další externí soubory (obráz\-ky, textové soubory atd.)}
}


\newglossaryentry{new}{
	name={new},
	description={\hspace{0.1mm} Výraz uvozující vytvoření instance z objektu}
}
\newglossaryentry{strokeWeight()}{
	name={strokeWeight()},
	description={Nastavuje tloušťku čáry u kresby kontur objektů. Základní nastavení je 1.0}
}
\newglossaryentry{loadStrings()}{
	name={loadStrings()},
	description={Funkce načítající externí textová data do formátu pole řetězců: {\em String[]}. V takto vytvořeném poli každý slot obsahuje jeden řádek textu}
}
\newglossaryentry{splitTokens()}{
	name={splitTokens()},
	description={Funkce pro práci s textem přijímající dva parametry: samotný řetězec a dále takzvaný delimiter -- znak, kterým řetězec chce\-me rozdělit. Funkce vrací pole řetězců, {\em String[]}}
}
\newglossaryentry{import}{
	name={import},
	description={Výraz načítá externí součásti do {\em Processingu}. Využívá se zejména při práci s knihovnami}
}